% Options for packages loaded elsewhere
\PassOptionsToPackage{unicode}{hyperref}
\PassOptionsToPackage{hyphens}{url}
%
\documentclass[
]{article}
\usepackage{amsmath,amssymb}
\usepackage{iftex}
\ifPDFTeX
  \usepackage[T1]{fontenc}
  \usepackage[utf8]{inputenc}
  \usepackage{textcomp} % provide euro and other symbols
\else % if luatex or xetex
  \usepackage{unicode-math} % this also loads fontspec
  \defaultfontfeatures{Scale=MatchLowercase}
  \defaultfontfeatures[\rmfamily]{Ligatures=TeX,Scale=1}
\fi
\usepackage{lmodern}
\ifPDFTeX\else
  % xetex/luatex font selection
\fi
% Use upquote if available, for straight quotes in verbatim environments
\IfFileExists{upquote.sty}{\usepackage{upquote}}{}
\IfFileExists{microtype.sty}{% use microtype if available
  \usepackage[]{microtype}
  \UseMicrotypeSet[protrusion]{basicmath} % disable protrusion for tt fonts
}{}
\makeatletter
\@ifundefined{KOMAClassName}{% if non-KOMA class
  \IfFileExists{parskip.sty}{%
    \usepackage{parskip}
  }{% else
    \setlength{\parindent}{0pt}
    \setlength{\parskip}{6pt plus 2pt minus 1pt}}
}{% if KOMA class
  \KOMAoptions{parskip=half}}
\makeatother
\usepackage{xcolor}
\usepackage[margin=1in]{geometry}
\usepackage{color}
\usepackage{fancyvrb}
\newcommand{\VerbBar}{|}
\newcommand{\VERB}{\Verb[commandchars=\\\{\}]}
\DefineVerbatimEnvironment{Highlighting}{Verbatim}{commandchars=\\\{\}}
% Add ',fontsize=\small' for more characters per line
\usepackage{framed}
\definecolor{shadecolor}{RGB}{248,248,248}
\newenvironment{Shaded}{\begin{snugshade}}{\end{snugshade}}
\newcommand{\AlertTok}[1]{\textcolor[rgb]{0.94,0.16,0.16}{#1}}
\newcommand{\AnnotationTok}[1]{\textcolor[rgb]{0.56,0.35,0.01}{\textbf{\textit{#1}}}}
\newcommand{\AttributeTok}[1]{\textcolor[rgb]{0.13,0.29,0.53}{#1}}
\newcommand{\BaseNTok}[1]{\textcolor[rgb]{0.00,0.00,0.81}{#1}}
\newcommand{\BuiltInTok}[1]{#1}
\newcommand{\CharTok}[1]{\textcolor[rgb]{0.31,0.60,0.02}{#1}}
\newcommand{\CommentTok}[1]{\textcolor[rgb]{0.56,0.35,0.01}{\textit{#1}}}
\newcommand{\CommentVarTok}[1]{\textcolor[rgb]{0.56,0.35,0.01}{\textbf{\textit{#1}}}}
\newcommand{\ConstantTok}[1]{\textcolor[rgb]{0.56,0.35,0.01}{#1}}
\newcommand{\ControlFlowTok}[1]{\textcolor[rgb]{0.13,0.29,0.53}{\textbf{#1}}}
\newcommand{\DataTypeTok}[1]{\textcolor[rgb]{0.13,0.29,0.53}{#1}}
\newcommand{\DecValTok}[1]{\textcolor[rgb]{0.00,0.00,0.81}{#1}}
\newcommand{\DocumentationTok}[1]{\textcolor[rgb]{0.56,0.35,0.01}{\textbf{\textit{#1}}}}
\newcommand{\ErrorTok}[1]{\textcolor[rgb]{0.64,0.00,0.00}{\textbf{#1}}}
\newcommand{\ExtensionTok}[1]{#1}
\newcommand{\FloatTok}[1]{\textcolor[rgb]{0.00,0.00,0.81}{#1}}
\newcommand{\FunctionTok}[1]{\textcolor[rgb]{0.13,0.29,0.53}{\textbf{#1}}}
\newcommand{\ImportTok}[1]{#1}
\newcommand{\InformationTok}[1]{\textcolor[rgb]{0.56,0.35,0.01}{\textbf{\textit{#1}}}}
\newcommand{\KeywordTok}[1]{\textcolor[rgb]{0.13,0.29,0.53}{\textbf{#1}}}
\newcommand{\NormalTok}[1]{#1}
\newcommand{\OperatorTok}[1]{\textcolor[rgb]{0.81,0.36,0.00}{\textbf{#1}}}
\newcommand{\OtherTok}[1]{\textcolor[rgb]{0.56,0.35,0.01}{#1}}
\newcommand{\PreprocessorTok}[1]{\textcolor[rgb]{0.56,0.35,0.01}{\textit{#1}}}
\newcommand{\RegionMarkerTok}[1]{#1}
\newcommand{\SpecialCharTok}[1]{\textcolor[rgb]{0.81,0.36,0.00}{\textbf{#1}}}
\newcommand{\SpecialStringTok}[1]{\textcolor[rgb]{0.31,0.60,0.02}{#1}}
\newcommand{\StringTok}[1]{\textcolor[rgb]{0.31,0.60,0.02}{#1}}
\newcommand{\VariableTok}[1]{\textcolor[rgb]{0.00,0.00,0.00}{#1}}
\newcommand{\VerbatimStringTok}[1]{\textcolor[rgb]{0.31,0.60,0.02}{#1}}
\newcommand{\WarningTok}[1]{\textcolor[rgb]{0.56,0.35,0.01}{\textbf{\textit{#1}}}}
\usepackage{longtable,booktabs,array}
\usepackage{calc} % for calculating minipage widths
% Correct order of tables after \paragraph or \subparagraph
\usepackage{etoolbox}
\makeatletter
\patchcmd\longtable{\par}{\if@noskipsec\mbox{}\fi\par}{}{}
\makeatother
% Allow footnotes in longtable head/foot
\IfFileExists{footnotehyper.sty}{\usepackage{footnotehyper}}{\usepackage{footnote}}
\makesavenoteenv{longtable}
\usepackage{graphicx}
\makeatletter
\def\maxwidth{\ifdim\Gin@nat@width>\linewidth\linewidth\else\Gin@nat@width\fi}
\def\maxheight{\ifdim\Gin@nat@height>\textheight\textheight\else\Gin@nat@height\fi}
\makeatother
% Scale images if necessary, so that they will not overflow the page
% margins by default, and it is still possible to overwrite the defaults
% using explicit options in \includegraphics[width, height, ...]{}
\setkeys{Gin}{width=\maxwidth,height=\maxheight,keepaspectratio}
% Set default figure placement to htbp
\makeatletter
\def\fps@figure{htbp}
\makeatother
\setlength{\emergencystretch}{3em} % prevent overfull lines
\providecommand{\tightlist}{%
  \setlength{\itemsep}{0pt}\setlength{\parskip}{0pt}}
\setcounter{secnumdepth}{-\maxdimen} % remove section numbering
\ifLuaTeX
  \usepackage{selnolig}  % disable illegal ligatures
\fi
\usepackage{bookmark}
\IfFileExists{xurl.sty}{\usepackage{xurl}}{} % add URL line breaks if available
\urlstyle{same}
\hypersetup{
  pdftitle={Báo cáo Phân tích Thu nhập của Freelancer},
  pdfauthor={Nhóm {[}Tên Nhóm Của Bạn{]}},
  hidelinks,
  pdfcreator={LaTeX via pandoc}}

\title{Báo cáo Phân tích Thu nhập của Freelancer}
\author{Nhóm {[}Tên Nhóm Của Bạn{]}}
\date{2025-05-16}

\begin{document}
\maketitle

{
\setcounter{tocdepth}{2}
\tableofcontents
}
\section{1. Giới thiệu}\label{giux1edbi-thiux1ec7u}

Báo cáo này nhằm mục đích phân tích chi tiết tập dữ liệu về thu nhập của
các freelancer. Chúng tôi sẽ khám phá các yếu tố như loại công việc, nền
tảng, mức độ kinh nghiệm, khu vực khách hàng, và các chỉ số hiệu suất
khác ảnh hưởng như thế nào đến thu nhập (Earnings\_USD) của freelancer.
Thông qua phân tích dữ liệu khám phá (EDA) và xây dựng các mô hình dự
đoán, chúng tôi hy vọng sẽ rút ra được những hiểu biết sâu sắc và có giá
trị, giúp các freelancer tối ưu hóa chiến lược làm việc và thu nhập của
mình. Dữ liệu được sử dụng là freelancer\_earnings\_bd.csv, chứa thông
tin đa dạng về các freelancer và công việc của họ.

\section{2. Tải và Chuẩn bị Dữ
liệu}\label{tux1ea3i-vuxe0-chuux1ea9n-bux1ecb-dux1eef-liux1ec7u}

\subsection{2.1. Tải Dữ liệu
Thô}\label{tux1ea3i-dux1eef-liux1ec7u-thuxf4}

\begin{Shaded}
\begin{Highlighting}[]
\CommentTok{\# Đặt đường dẫn làm việc (working directory) nếu cần, hoặc sử dụng RStudio Project}
\CommentTok{\# Ví dụ: setwd("D:/Path/To/Your/ProjectFolder") }
\CommentTok{\# Giả sử file CSV nằm trong cùng thư mục với file .Rmd hoặc trong thư mục con \textquotesingle{}data\textquotesingle{}}
\CommentTok{\# Hãy đảm bảo đường dẫn file CSV là chính xác}
\NormalTok{data\_raw }\OtherTok{\textless{}{-}} \FunctionTok{read\_csv}\NormalTok{(}\StringTok{"data/freelancer\_earnings\_bd.csv"}\NormalTok{) }

\CommentTok{\# Hiển thị vài dòng đầu của dữ liệu thô}
\FunctionTok{kable}\NormalTok{(}\FunctionTok{head}\NormalTok{(data\_raw), }\AttributeTok{caption =} \StringTok{"Dữ liệu thô ban đầu (6 dòng đầu)"}\NormalTok{)}
\end{Highlighting}
\end{Shaded}

\begin{longtable}[]{@{}
  >{\raggedleft\arraybackslash}p{(\columnwidth - 28\tabcolsep) * \real{0.0633}}
  >{\raggedright\arraybackslash}p{(\columnwidth - 28\tabcolsep) * \real{0.0814}}
  >{\raggedright\arraybackslash}p{(\columnwidth - 28\tabcolsep) * \real{0.0633}}
  >{\raggedright\arraybackslash}p{(\columnwidth - 28\tabcolsep) * \real{0.0769}}
  >{\raggedright\arraybackslash}p{(\columnwidth - 28\tabcolsep) * \real{0.0633}}
  >{\raggedright\arraybackslash}p{(\columnwidth - 28\tabcolsep) * \real{0.0679}}
  >{\raggedleft\arraybackslash}p{(\columnwidth - 28\tabcolsep) * \real{0.0633}}
  >{\raggedleft\arraybackslash}p{(\columnwidth - 28\tabcolsep) * \real{0.0588}}
  >{\raggedleft\arraybackslash}p{(\columnwidth - 28\tabcolsep) * \real{0.0543}}
  >{\raggedleft\arraybackslash}p{(\columnwidth - 28\tabcolsep) * \real{0.0769}}
  >{\raggedleft\arraybackslash}p{(\columnwidth - 28\tabcolsep) * \real{0.0633}}
  >{\raggedleft\arraybackslash}p{(\columnwidth - 28\tabcolsep) * \real{0.0814}}
  >{\raggedright\arraybackslash}p{(\columnwidth - 28\tabcolsep) * \real{0.0588}}
  >{\raggedleft\arraybackslash}p{(\columnwidth - 28\tabcolsep) * \real{0.0543}}
  >{\raggedleft\arraybackslash}p{(\columnwidth - 28\tabcolsep) * \real{0.0724}}@{}}
\caption{Dữ liệu thô ban đầu (6 dòng đầu)}\tabularnewline
\toprule\noalign{}
\begin{minipage}[b]{\linewidth}\raggedleft
Freelancer\_ID
\end{minipage} & \begin{minipage}[b]{\linewidth}\raggedright
Job\_Category
\end{minipage} & \begin{minipage}[b]{\linewidth}\raggedright
Platform
\end{minipage} & \begin{minipage}[b]{\linewidth}\raggedright
Experience\_Level
\end{minipage} & \begin{minipage}[b]{\linewidth}\raggedright
Client\_Region
\end{minipage} & \begin{minipage}[b]{\linewidth}\raggedright
Payment\_Method
\end{minipage} & \begin{minipage}[b]{\linewidth}\raggedleft
Job\_Completed
\end{minipage} & \begin{minipage}[b]{\linewidth}\raggedleft
Earnings\_USD
\end{minipage} & \begin{minipage}[b]{\linewidth}\raggedleft
Hourly\_Rate
\end{minipage} & \begin{minipage}[b]{\linewidth}\raggedleft
Job\_Success\_Rate
\end{minipage} & \begin{minipage}[b]{\linewidth}\raggedleft
Client\_Rating
\end{minipage} & \begin{minipage}[b]{\linewidth}\raggedleft
Job\_Duration\_Days
\end{minipage} & \begin{minipage}[b]{\linewidth}\raggedright
Project\_Type
\end{minipage} & \begin{minipage}[b]{\linewidth}\raggedleft
Rehire\_Rate
\end{minipage} & \begin{minipage}[b]{\linewidth}\raggedleft
Marketing\_Spend
\end{minipage} \\
\midrule\noalign{}
\endfirsthead
\toprule\noalign{}
\begin{minipage}[b]{\linewidth}\raggedleft
Freelancer\_ID
\end{minipage} & \begin{minipage}[b]{\linewidth}\raggedright
Job\_Category
\end{minipage} & \begin{minipage}[b]{\linewidth}\raggedright
Platform
\end{minipage} & \begin{minipage}[b]{\linewidth}\raggedright
Experience\_Level
\end{minipage} & \begin{minipage}[b]{\linewidth}\raggedright
Client\_Region
\end{minipage} & \begin{minipage}[b]{\linewidth}\raggedright
Payment\_Method
\end{minipage} & \begin{minipage}[b]{\linewidth}\raggedleft
Job\_Completed
\end{minipage} & \begin{minipage}[b]{\linewidth}\raggedleft
Earnings\_USD
\end{minipage} & \begin{minipage}[b]{\linewidth}\raggedleft
Hourly\_Rate
\end{minipage} & \begin{minipage}[b]{\linewidth}\raggedleft
Job\_Success\_Rate
\end{minipage} & \begin{minipage}[b]{\linewidth}\raggedleft
Client\_Rating
\end{minipage} & \begin{minipage}[b]{\linewidth}\raggedleft
Job\_Duration\_Days
\end{minipage} & \begin{minipage}[b]{\linewidth}\raggedright
Project\_Type
\end{minipage} & \begin{minipage}[b]{\linewidth}\raggedleft
Rehire\_Rate
\end{minipage} & \begin{minipage}[b]{\linewidth}\raggedleft
Marketing\_Spend
\end{minipage} \\
\midrule\noalign{}
\endhead
\bottomrule\noalign{}
\endlastfoot
1 & Web Development & Fiverr & Beginner & Asia & Mobile Banking & 180 &
1620 & 95.79 & 68.73 & 3.18 & 1 & Fixed & 40.19 & 53 \\
2 & App Development & Fiverr & Beginner & Australia & Mobile Banking &
218 & 9078 & 86.38 & 97.54 & 3.44 & 54 & Fixed & 36.53 & 486 \\
3 & Web Development & Fiverr & Beginner & UK & Crypto & 27 & 3455 &
85.17 & 86.60 & 4.20 & 46 & Hourly & 74.05 & 489 \\
4 & Data Entry & PeoplePerHour & Intermediate & Asia & Bank Transfer &
17 & 5577 & 14.37 & 79.93 & 4.47 & 41 & Hourly & 27.58 & 67 \\
5 & Digital Marketing & Upwork & Expert & Asia & Crypto & 245 & 5898 &
99.37 & 57.80 & 5.00 & 41 & Hourly & 69.09 & 489 \\
6 & Customer Support & Toptal & Beginner & Europe & Crypto & 280 & 6867
& 43.04 & 57.80 & 4.87 & 8 & Fixed & 43.88 & 290 \\
\end{longtable}

\begin{Shaded}
\begin{Highlighting}[]
\CommentTok{\# Xem cấu trúc dữ liệu thô}
\FunctionTok{str}\NormalTok{(data\_raw, }\AttributeTok{list.len=}\FunctionTok{ncol}\NormalTok{(data\_raw))}
\end{Highlighting}
\end{Shaded}

\begin{verbatim}
## spc_tbl_ [1,950 x 15] (S3: spec_tbl_df/tbl_df/tbl/data.frame)
##  $ Freelancer_ID    : num [1:1950] 1 2 3 4 5 6 7 8 9 10 ...
##  $ Job_Category     : chr [1:1950] "Web Development" "App Development" "Web Development" "Data Entry" ...
##  $ Platform         : chr [1:1950] "Fiverr" "Fiverr" "Fiverr" "PeoplePerHour" ...
##  $ Experience_Level : chr [1:1950] "Beginner" "Beginner" "Beginner" "Intermediate" ...
##  $ Client_Region    : chr [1:1950] "Asia" "Australia" "UK" "Asia" ...
##  $ Payment_Method   : chr [1:1950] "Mobile Banking" "Mobile Banking" "Crypto" "Bank Transfer" ...
##  $ Job_Completed    : num [1:1950] 180 218 27 17 245 280 96 112 233 156 ...
##  $ Earnings_USD     : num [1:1950] 1620 9078 3455 5577 5898 ...
##  $ Hourly_Rate      : num [1:1950] 95.8 86.4 85.2 14.4 99.4 ...
##  $ Job_Success_Rate : num [1:1950] 68.7 97.5 86.6 79.9 57.8 ...
##  $ Client_Rating    : num [1:1950] 3.18 3.44 4.2 4.47 5 4.87 4.29 3.84 4.27 4.57 ...
##  $ Job_Duration_Days: num [1:1950] 1 54 46 41 41 8 32 30 46 52 ...
##  $ Project_Type     : chr [1:1950] "Fixed" "Fixed" "Hourly" "Hourly" ...
##  $ Rehire_Rate      : num [1:1950] 40.2 36.5 74 27.6 69.1 ...
##  $ Marketing_Spend  : num [1:1950] 53 486 489 67 489 290 343 168 396 160 ...
##  - attr(*, "spec")=
##   .. cols(
##   ..   Freelancer_ID = col_double(),
##   ..   Job_Category = col_character(),
##   ..   Platform = col_character(),
##   ..   Experience_Level = col_character(),
##   ..   Client_Region = col_character(),
##   ..   Payment_Method = col_character(),
##   ..   Job_Completed = col_double(),
##   ..   Earnings_USD = col_double(),
##   ..   Hourly_Rate = col_double(),
##   ..   Job_Success_Rate = col_double(),
##   ..   Client_Rating = col_double(),
##   ..   Job_Duration_Days = col_double(),
##   ..   Project_Type = col_character(),
##   ..   Rehire_Rate = col_double(),
##   ..   Marketing_Spend = col_double()
##   .. )
##  - attr(*, "problems")=<externalptr>
\end{verbatim}

\subsection{2.2. Làm sạch Dữ
liệu}\label{luxe0m-sux1ea1ch-dux1eef-liux1ec7u}

Dựa trên code backend đã có, chúng ta thực hiện các bước làm sạch sau:

\begin{itemize}
\item
  Loại bỏ các dòng có Earnings\_USD bị thiếu.
\item
  Chuyển đổi kiểu dữ liệu cho các cột số.
\item
  Chuẩn hóa tên các giá trị trong các cột dạng ký tự (Job\_Category,
  Experience\_Level, etc.) sang dạng Title Case.
\item
  Đổi tên cột cho nhất quán và dễ sử dụng hơn.
\end{itemize}

\begin{Shaded}
\begin{Highlighting}[]
\NormalTok{data\_clean }\OtherTok{\textless{}{-}}\NormalTok{ data\_raw }\SpecialCharTok{\%\textgreater{}\%}
  \FunctionTok{filter}\NormalTok{(}\SpecialCharTok{!}\FunctionTok{is.na}\NormalTok{(Earnings\_USD)) }\SpecialCharTok{\%\textgreater{}\%} \CommentTok{\# Loại bỏ NA trong cột thu nhập chính}
  \FunctionTok{mutate}\NormalTok{(}
    \CommentTok{\# Đổi tên và chuyển đổi kiểu dữ liệu}
    \AttributeTok{FreelancerID =}\NormalTok{ Freelancer\_ID, }\CommentTok{\# Giữ lại ID gốc nếu cần}
    \AttributeTok{JobCategory =} \FunctionTok{str\_to\_title}\NormalTok{(Job\_Category),}
    \AttributeTok{Platform =} \FunctionTok{str\_to\_title}\NormalTok{(Platform),}
    \AttributeTok{ExperienceLevel =} \FunctionTok{factor}\NormalTok{(}\FunctionTok{str\_to\_title}\NormalTok{(Experience\_Level), }\AttributeTok{levels =} \FunctionTok{c}\NormalTok{(}\StringTok{"Beginner"}\NormalTok{, }\StringTok{"Intermediate"}\NormalTok{, }\StringTok{"Expert"}\NormalTok{), }\AttributeTok{ordered =} \ConstantTok{TRUE}\NormalTok{), }\CommentTok{\# Chuyển thành factor có thứ tự}
    \AttributeTok{ClientRegion =} \FunctionTok{str\_to\_title}\NormalTok{(Client\_Region),}
    \AttributeTok{PaymentMethod =} \FunctionTok{str\_to\_title}\NormalTok{(Payment\_Method),}
    \AttributeTok{JobsCompleted =} \FunctionTok{as.numeric}\NormalTok{(Job\_Completed),}
    \AttributeTok{EarningsUSD =} \FunctionTok{as.numeric}\NormalTok{(Earnings\_USD), }\CommentTok{\# Đổi tên từ Income để rõ ràng}
    \AttributeTok{HourlyRate =} \FunctionTok{as.numeric}\NormalTok{(Hourly\_Rate),}
    \AttributeTok{JobSuccessRate =} \FunctionTok{as.numeric}\NormalTok{(Job\_Success\_Rate),}
    \AttributeTok{ClientRating =} \FunctionTok{as.numeric}\NormalTok{(Client\_Rating),}
    \AttributeTok{JobDurationDays =} \FunctionTok{as.numeric}\NormalTok{(Job\_Duration\_Days),}
    \AttributeTok{ProjectType =} \FunctionTok{factor}\NormalTok{(}\FunctionTok{str\_to\_title}\NormalTok{(Project\_Type)), }\CommentTok{\# Chuyển thành factor}
    \AttributeTok{RehireRate =} \FunctionTok{as.numeric}\NormalTok{(Rehire\_Rate),}
    \AttributeTok{MarketingSpend =} \FunctionTok{as.numeric}\NormalTok{(Marketing\_Spend)}
\NormalTok{  ) }\SpecialCharTok{\%\textgreater{}\%}
  \CommentTok{\# Chọn các cột đã được làm sạch và đổi tên}
  \FunctionTok{select}\NormalTok{(}
\NormalTok{    FreelancerID, JobCategory, Platform, ExperienceLevel, ClientRegion, PaymentMethod,}
\NormalTok{    JobsCompleted, EarningsUSD, HourlyRate, JobSuccessRate, ClientRating,}
\NormalTok{    JobDurationDays, ProjectType, RehireRate, MarketingSpend}
\NormalTok{  )}

\CommentTok{\# Kiểm tra lại các giá trị NA sau khi làm sạch cơ bản}
\FunctionTok{colSums}\NormalTok{(}\FunctionTok{is.na}\NormalTok{(data\_clean)) }\SpecialCharTok{\%\textgreater{}\%} \FunctionTok{kable}\NormalTok{(}\AttributeTok{col.names =} \FunctionTok{c}\NormalTok{(}\StringTok{"Số lượng NA"}\NormalTok{), }\AttributeTok{caption =} \StringTok{"Kiểm tra NA sau làm sạch"}\NormalTok{)}
\end{Highlighting}
\end{Shaded}

\begin{longtable}[]{@{}lr@{}}
\caption{Kiểm tra NA sau làm sạch}\tabularnewline
\toprule\noalign{}
& Số lượng NA \\
\midrule\noalign{}
\endfirsthead
\toprule\noalign{}
& Số lượng NA \\
\midrule\noalign{}
\endhead
\bottomrule\noalign{}
\endlastfoot
FreelancerID & 0 \\
JobCategory & 0 \\
Platform & 0 \\
ExperienceLevel & 0 \\
ClientRegion & 0 \\
PaymentMethod & 0 \\
JobsCompleted & 0 \\
EarningsUSD & 0 \\
HourlyRate & 0 \\
JobSuccessRate & 0 \\
ClientRating & 0 \\
JobDurationDays & 0 \\
ProjectType & 0 \\
RehireRate & 0 \\
MarketingSpend & 0 \\
\end{longtable}

\begin{Shaded}
\begin{Highlighting}[]
\CommentTok{\# Hiển thị vài dòng đầu của dữ liệu đã làm sạch}
\FunctionTok{kable}\NormalTok{(}\FunctionTok{head}\NormalTok{(data\_clean), }\AttributeTok{caption =} \StringTok{"Dữ liệu đã làm sạch (6 dòng đầu)"}\NormalTok{)}
\end{Highlighting}
\end{Shaded}

\begin{longtable}[]{@{}
  >{\raggedleft\arraybackslash}p{(\columnwidth - 28\tabcolsep) * \real{0.0625}}
  >{\raggedright\arraybackslash}p{(\columnwidth - 28\tabcolsep) * \real{0.0865}}
  >{\raggedright\arraybackslash}p{(\columnwidth - 28\tabcolsep) * \real{0.0673}}
  >{\raggedright\arraybackslash}p{(\columnwidth - 28\tabcolsep) * \real{0.0769}}
  >{\raggedright\arraybackslash}p{(\columnwidth - 28\tabcolsep) * \real{0.0625}}
  >{\raggedright\arraybackslash}p{(\columnwidth - 28\tabcolsep) * \real{0.0721}}
  >{\raggedleft\arraybackslash}p{(\columnwidth - 28\tabcolsep) * \real{0.0673}}
  >{\raggedleft\arraybackslash}p{(\columnwidth - 28\tabcolsep) * \real{0.0577}}
  >{\raggedleft\arraybackslash}p{(\columnwidth - 28\tabcolsep) * \real{0.0529}}
  >{\raggedleft\arraybackslash}p{(\columnwidth - 28\tabcolsep) * \real{0.0721}}
  >{\raggedleft\arraybackslash}p{(\columnwidth - 28\tabcolsep) * \real{0.0625}}
  >{\raggedleft\arraybackslash}p{(\columnwidth - 28\tabcolsep) * \real{0.0769}}
  >{\raggedright\arraybackslash}p{(\columnwidth - 28\tabcolsep) * \real{0.0577}}
  >{\raggedleft\arraybackslash}p{(\columnwidth - 28\tabcolsep) * \real{0.0529}}
  >{\raggedleft\arraybackslash}p{(\columnwidth - 28\tabcolsep) * \real{0.0721}}@{}}
\caption{Dữ liệu đã làm sạch (6 dòng đầu)}\tabularnewline
\toprule\noalign{}
\begin{minipage}[b]{\linewidth}\raggedleft
FreelancerID
\end{minipage} & \begin{minipage}[b]{\linewidth}\raggedright
JobCategory
\end{minipage} & \begin{minipage}[b]{\linewidth}\raggedright
Platform
\end{minipage} & \begin{minipage}[b]{\linewidth}\raggedright
ExperienceLevel
\end{minipage} & \begin{minipage}[b]{\linewidth}\raggedright
ClientRegion
\end{minipage} & \begin{minipage}[b]{\linewidth}\raggedright
PaymentMethod
\end{minipage} & \begin{minipage}[b]{\linewidth}\raggedleft
JobsCompleted
\end{minipage} & \begin{minipage}[b]{\linewidth}\raggedleft
EarningsUSD
\end{minipage} & \begin{minipage}[b]{\linewidth}\raggedleft
HourlyRate
\end{minipage} & \begin{minipage}[b]{\linewidth}\raggedleft
JobSuccessRate
\end{minipage} & \begin{minipage}[b]{\linewidth}\raggedleft
ClientRating
\end{minipage} & \begin{minipage}[b]{\linewidth}\raggedleft
JobDurationDays
\end{minipage} & \begin{minipage}[b]{\linewidth}\raggedright
ProjectType
\end{minipage} & \begin{minipage}[b]{\linewidth}\raggedleft
RehireRate
\end{minipage} & \begin{minipage}[b]{\linewidth}\raggedleft
MarketingSpend
\end{minipage} \\
\midrule\noalign{}
\endfirsthead
\toprule\noalign{}
\begin{minipage}[b]{\linewidth}\raggedleft
FreelancerID
\end{minipage} & \begin{minipage}[b]{\linewidth}\raggedright
JobCategory
\end{minipage} & \begin{minipage}[b]{\linewidth}\raggedright
Platform
\end{minipage} & \begin{minipage}[b]{\linewidth}\raggedright
ExperienceLevel
\end{minipage} & \begin{minipage}[b]{\linewidth}\raggedright
ClientRegion
\end{minipage} & \begin{minipage}[b]{\linewidth}\raggedright
PaymentMethod
\end{minipage} & \begin{minipage}[b]{\linewidth}\raggedleft
JobsCompleted
\end{minipage} & \begin{minipage}[b]{\linewidth}\raggedleft
EarningsUSD
\end{minipage} & \begin{minipage}[b]{\linewidth}\raggedleft
HourlyRate
\end{minipage} & \begin{minipage}[b]{\linewidth}\raggedleft
JobSuccessRate
\end{minipage} & \begin{minipage}[b]{\linewidth}\raggedleft
ClientRating
\end{minipage} & \begin{minipage}[b]{\linewidth}\raggedleft
JobDurationDays
\end{minipage} & \begin{minipage}[b]{\linewidth}\raggedright
ProjectType
\end{minipage} & \begin{minipage}[b]{\linewidth}\raggedleft
RehireRate
\end{minipage} & \begin{minipage}[b]{\linewidth}\raggedleft
MarketingSpend
\end{minipage} \\
\midrule\noalign{}
\endhead
\bottomrule\noalign{}
\endlastfoot
1 & Web Development & Fiverr & Beginner & Asia & Mobile Banking & 180 &
1620 & 95.79 & 68.73 & 3.18 & 1 & Fixed & 40.19 & 53 \\
2 & App Development & Fiverr & Beginner & Australia & Mobile Banking &
218 & 9078 & 86.38 & 97.54 & 3.44 & 54 & Fixed & 36.53 & 486 \\
3 & Web Development & Fiverr & Beginner & Uk & Crypto & 27 & 3455 &
85.17 & 86.60 & 4.20 & 46 & Hourly & 74.05 & 489 \\
4 & Data Entry & Peopleperhour & Intermediate & Asia & Bank Transfer &
17 & 5577 & 14.37 & 79.93 & 4.47 & 41 & Hourly & 27.58 & 67 \\
5 & Digital Marketing & Upwork & Expert & Asia & Crypto & 245 & 5898 &
99.37 & 57.80 & 5.00 & 41 & Hourly & 69.09 & 489 \\
6 & Customer Support & Toptal & Beginner & Europe & Crypto & 280 & 6867
& 43.04 & 57.80 & 4.87 & 8 & Fixed & 43.88 & 290 \\
\end{longtable}

\begin{Shaded}
\begin{Highlighting}[]
\CommentTok{\# Xem cấu trúc dữ liệu đã làm sạch}
\FunctionTok{str}\NormalTok{(data\_clean, }\AttributeTok{list.len=}\FunctionTok{ncol}\NormalTok{(data\_clean))}
\end{Highlighting}
\end{Shaded}

\begin{verbatim}
## tibble [1,950 x 15] (S3: tbl_df/tbl/data.frame)
##  $ FreelancerID   : num [1:1950] 1 2 3 4 5 6 7 8 9 10 ...
##  $ JobCategory    : chr [1:1950] "Web Development" "App Development" "Web Development" "Data Entry" ...
##  $ Platform       : chr [1:1950] "Fiverr" "Fiverr" "Fiverr" "Peopleperhour" ...
##  $ ExperienceLevel: Ord.factor w/ 3 levels "Beginner"<"Intermediate"<..: 1 1 1 2 3 1 1 1 2 1 ...
##  $ ClientRegion   : chr [1:1950] "Asia" "Australia" "Uk" "Asia" ...
##  $ PaymentMethod  : chr [1:1950] "Mobile Banking" "Mobile Banking" "Crypto" "Bank Transfer" ...
##  $ JobsCompleted  : num [1:1950] 180 218 27 17 245 280 96 112 233 156 ...
##  $ EarningsUSD    : num [1:1950] 1620 9078 3455 5577 5898 ...
##  $ HourlyRate     : num [1:1950] 95.8 86.4 85.2 14.4 99.4 ...
##  $ JobSuccessRate : num [1:1950] 68.7 97.5 86.6 79.9 57.8 ...
##  $ ClientRating   : num [1:1950] 3.18 3.44 4.2 4.47 5 4.87 4.29 3.84 4.27 4.57 ...
##  $ JobDurationDays: num [1:1950] 1 54 46 41 41 8 32 30 46 52 ...
##  $ ProjectType    : Factor w/ 2 levels "Fixed","Hourly": 1 1 2 2 2 1 2 1 1 2 ...
##  $ RehireRate     : num [1:1950] 40.2 36.5 74 27.6 69.1 ...
##  $ MarketingSpend : num [1:1950] 53 486 489 67 489 290 343 168 396 160 ...
\end{verbatim}

\textbf{Nhận xét:}~Sau bước làm sạch cơ bản, chúng ta không còn giá trị
NA nào trong các cột đã chọn (hoặc số lượng NA rất ít nếu có trong các
cột khác). Kiểu dữ liệu đã được chuẩn hóa, và các cột categorical quan
trọng như~ExperienceLevel~và~ProjectType~đã được chuyển thành factor.

\subsection{2.3. Chuẩn hóa Dữ liệu
(Normalization)}\label{chuux1ea9n-huxf3a-dux1eef-liux1ec7u-normalization}

Chuẩn hóa Min-Max đưa tất cả các giá trị về khoảng {[}0, 1{]}. Điều này
có thể hữu ích cho một số thuật toán học máy nhạy cảm với thang đo của
dữ liệu hoặc khi muốn so sánh các biến có đơn vị khác nhau trên cùng một
biểu đồ.

\begin{Shaded}
\begin{Highlighting}[]
\NormalTok{data\_norm }\OtherTok{\textless{}{-}}\NormalTok{ data\_clean }\SpecialCharTok{\%\textgreater{}\%}
  \FunctionTok{mutate}\NormalTok{(}
    \AttributeTok{EarningsUSD\_norm =}\NormalTok{ (EarningsUSD }\SpecialCharTok{{-}} \FunctionTok{min}\NormalTok{(EarningsUSD, }\AttributeTok{na.rm =} \ConstantTok{TRUE}\NormalTok{)) }\SpecialCharTok{/}\NormalTok{ (}\FunctionTok{max}\NormalTok{(EarningsUSD, }\AttributeTok{na.rm =} \ConstantTok{TRUE}\NormalTok{) }\SpecialCharTok{{-}} \FunctionTok{min}\NormalTok{(EarningsUSD, }\AttributeTok{na.rm =} \ConstantTok{TRUE}\NormalTok{)),}
    \AttributeTok{JobSuccessRate\_norm =}\NormalTok{ (JobSuccessRate }\SpecialCharTok{{-}} \FunctionTok{min}\NormalTok{(JobSuccessRate, }\AttributeTok{na.rm =} \ConstantTok{TRUE}\NormalTok{)) }\SpecialCharTok{/}\NormalTok{ (}\FunctionTok{max}\NormalTok{(JobSuccessRate, }\AttributeTok{na.rm =} \ConstantTok{TRUE}\NormalTok{) }\SpecialCharTok{{-}} \FunctionTok{min}\NormalTok{(JobSuccessRate, }\AttributeTok{na.rm =} \ConstantTok{TRUE}\NormalTok{)),}
    \AttributeTok{ClientRating\_norm =}\NormalTok{ (ClientRating }\SpecialCharTok{{-}} \FunctionTok{min}\NormalTok{(ClientRating, }\AttributeTok{na.rm =} \ConstantTok{TRUE}\NormalTok{)) }\SpecialCharTok{/}\NormalTok{ (}\FunctionTok{max}\NormalTok{(ClientRating, }\AttributeTok{na.rm =} \ConstantTok{TRUE}\NormalTok{) }\SpecialCharTok{{-}} \FunctionTok{min}\NormalTok{(ClientRating, }\AttributeTok{na.rm =} \ConstantTok{TRUE}\NormalTok{)),}
    \AttributeTok{HourlyRate\_norm =}\NormalTok{ (HourlyRate }\SpecialCharTok{{-}} \FunctionTok{min}\NormalTok{(HourlyRate, }\AttributeTok{na.rm =} \ConstantTok{TRUE}\NormalTok{)) }\SpecialCharTok{/}\NormalTok{ (}\FunctionTok{max}\NormalTok{(HourlyRate, }\AttributeTok{na.rm =} \ConstantTok{TRUE}\NormalTok{) }\SpecialCharTok{{-}} \FunctionTok{min}\NormalTok{(HourlyRate, }\AttributeTok{na.rm =} \ConstantTok{TRUE}\NormalTok{)),}
    \AttributeTok{JobsCompleted\_norm =}\NormalTok{ (JobsCompleted }\SpecialCharTok{{-}} \FunctionTok{min}\NormalTok{(JobsCompleted, }\AttributeTok{na.rm =} \ConstantTok{TRUE}\NormalTok{)) }\SpecialCharTok{/}\NormalTok{ (}\FunctionTok{max}\NormalTok{(JobsCompleted, }\AttributeTok{na.rm =} \ConstantTok{TRUE}\NormalTok{) }\SpecialCharTok{{-}} \FunctionTok{min}\NormalTok{(JobsCompleted, }\AttributeTok{na.rm =} \ConstantTok{TRUE}\NormalTok{))}
\NormalTok{  )}

\FunctionTok{kable}\NormalTok{(}\FunctionTok{head}\NormalTok{(data\_norm }\SpecialCharTok{\%\textgreater{}\%} \FunctionTok{select}\NormalTok{(EarningsUSD, EarningsUSD\_norm, JobSuccessRate, JobSuccessRate\_norm)), }
      \AttributeTok{caption =} \StringTok{"Ví dụ dữ liệu đã chuẩn hóa"}\NormalTok{)}
\end{Highlighting}
\end{Shaded}

\begin{longtable}[]{@{}rrrr@{}}
\caption{Ví dụ dữ liệu đã chuẩn hóa}\tabularnewline
\toprule\noalign{}
EarningsUSD & EarningsUSD\_norm & JobSuccessRate &
JobSuccessRate\_norm \\
\midrule\noalign{}
\endfirsthead
\toprule\noalign{}
EarningsUSD & EarningsUSD\_norm & JobSuccessRate &
JobSuccessRate\_norm \\
\midrule\noalign{}
\endhead
\bottomrule\noalign{}
\endlastfoot
1620 & 0.1578471 & 68.73 & 0.3726671 \\
9078 & 0.9081489 & 97.54 & 0.9508328 \\
3455 & 0.3424547 & 86.60 & 0.7312864 \\
5577 & 0.5559356 & 79.93 & 0.5974313 \\
5898 & 0.5882294 & 57.80 & 0.1533213 \\
6867 & 0.6857143 & 57.80 & 0.1533213 \\
\end{longtable}

\section{3. Phân tích Dữ liệu Khám phá
(EDA)}\label{phuxe2n-tuxedch-dux1eef-liux1ec7u-khuxe1m-phuxe1-eda}

\subsection{3.1. Thống kê Mô tả Tổng
quan}\label{thux1ed1ng-kuxea-muxf4-tux1ea3-tux1ed5ng-quan}

\begin{Shaded}
\begin{Highlighting}[]
\FunctionTok{skim}\NormalTok{(data\_clean)}
\end{Highlighting}
\end{Shaded}

\begin{longtable}[]{@{}ll@{}}
\caption{Data summary}\tabularnewline
\toprule\noalign{}
\endfirsthead
\endhead
\bottomrule\noalign{}
\endlastfoot
Name & data\_clean \\
Number of rows & 1950 \\
Number of columns & 15 \\
\_\_\_\_\_\_\_\_\_\_\_\_\_\_\_\_\_\_\_\_\_\_\_ & \\
Column type frequency: & \\
character & 4 \\
factor & 2 \\
numeric & 9 \\
\_\_\_\_\_\_\_\_\_\_\_\_\_\_\_\_\_\_\_\_\_\_\_\_ & \\
Group variables & None \\
\end{longtable}

\textbf{Variable type: character}

\begin{longtable}[]{@{}
  >{\raggedright\arraybackslash}p{(\columnwidth - 14\tabcolsep) * \real{0.1944}}
  >{\raggedleft\arraybackslash}p{(\columnwidth - 14\tabcolsep) * \real{0.1389}}
  >{\raggedleft\arraybackslash}p{(\columnwidth - 14\tabcolsep) * \real{0.1944}}
  >{\raggedleft\arraybackslash}p{(\columnwidth - 14\tabcolsep) * \real{0.0556}}
  >{\raggedleft\arraybackslash}p{(\columnwidth - 14\tabcolsep) * \real{0.0556}}
  >{\raggedleft\arraybackslash}p{(\columnwidth - 14\tabcolsep) * \real{0.0833}}
  >{\raggedleft\arraybackslash}p{(\columnwidth - 14\tabcolsep) * \real{0.1250}}
  >{\raggedleft\arraybackslash}p{(\columnwidth - 14\tabcolsep) * \real{0.1528}}@{}}
\toprule\noalign{}
\begin{minipage}[b]{\linewidth}\raggedright
skim\_variable
\end{minipage} & \begin{minipage}[b]{\linewidth}\raggedleft
n\_missing
\end{minipage} & \begin{minipage}[b]{\linewidth}\raggedleft
complete\_rate
\end{minipage} & \begin{minipage}[b]{\linewidth}\raggedleft
min
\end{minipage} & \begin{minipage}[b]{\linewidth}\raggedleft
max
\end{minipage} & \begin{minipage}[b]{\linewidth}\raggedleft
empty
\end{minipage} & \begin{minipage}[b]{\linewidth}\raggedleft
n\_unique
\end{minipage} & \begin{minipage}[b]{\linewidth}\raggedleft
whitespace
\end{minipage} \\
\midrule\noalign{}
\endhead
\bottomrule\noalign{}
\endlastfoot
JobCategory & 0 & 1 & 3 & 17 & 0 & 8 & 0 \\
Platform & 0 & 1 & 6 & 13 & 0 & 5 & 0 \\
ClientRegion & 0 & 1 & 2 & 11 & 0 & 7 & 0 \\
PaymentMethod & 0 & 1 & 6 & 14 & 0 & 4 & 0 \\
\end{longtable}

\textbf{Variable type: factor}

\begin{longtable}[]{@{}
  >{\raggedright\arraybackslash}p{(\columnwidth - 10\tabcolsep) * \real{0.1860}}
  >{\raggedleft\arraybackslash}p{(\columnwidth - 10\tabcolsep) * \real{0.1163}}
  >{\raggedleft\arraybackslash}p{(\columnwidth - 10\tabcolsep) * \real{0.1628}}
  >{\raggedright\arraybackslash}p{(\columnwidth - 10\tabcolsep) * \real{0.0930}}
  >{\raggedleft\arraybackslash}p{(\columnwidth - 10\tabcolsep) * \real{0.1047}}
  >{\raggedright\arraybackslash}p{(\columnwidth - 10\tabcolsep) * \real{0.3372}}@{}}
\toprule\noalign{}
\begin{minipage}[b]{\linewidth}\raggedright
skim\_variable
\end{minipage} & \begin{minipage}[b]{\linewidth}\raggedleft
n\_missing
\end{minipage} & \begin{minipage}[b]{\linewidth}\raggedleft
complete\_rate
\end{minipage} & \begin{minipage}[b]{\linewidth}\raggedright
ordered
\end{minipage} & \begin{minipage}[b]{\linewidth}\raggedleft
n\_unique
\end{minipage} & \begin{minipage}[b]{\linewidth}\raggedright
top\_counts
\end{minipage} \\
\midrule\noalign{}
\endhead
\bottomrule\noalign{}
\endlastfoot
ExperienceLevel & 0 & 1 & TRUE & 3 & Beg: 668, Int: 641, Exp: 641 \\
ProjectType & 0 & 1 & FALSE & 2 & Fix: 997, Hou: 953 \\
\end{longtable}

\textbf{Variable type: numeric}

\begin{longtable}[]{@{}
  >{\raggedright\arraybackslash}p{(\columnwidth - 20\tabcolsep) * \real{0.1600}}
  >{\raggedleft\arraybackslash}p{(\columnwidth - 20\tabcolsep) * \real{0.1000}}
  >{\raggedleft\arraybackslash}p{(\columnwidth - 20\tabcolsep) * \real{0.1400}}
  >{\raggedleft\arraybackslash}p{(\columnwidth - 20\tabcolsep) * \real{0.0800}}
  >{\raggedleft\arraybackslash}p{(\columnwidth - 20\tabcolsep) * \real{0.0800}}
  >{\raggedleft\arraybackslash}p{(\columnwidth - 20\tabcolsep) * \real{0.0600}}
  >{\raggedleft\arraybackslash}p{(\columnwidth - 20\tabcolsep) * \real{0.0800}}
  >{\raggedleft\arraybackslash}p{(\columnwidth - 20\tabcolsep) * \real{0.0800}}
  >{\raggedleft\arraybackslash}p{(\columnwidth - 20\tabcolsep) * \real{0.0800}}
  >{\raggedleft\arraybackslash}p{(\columnwidth - 20\tabcolsep) * \real{0.0800}}
  >{\raggedright\arraybackslash}p{(\columnwidth - 20\tabcolsep) * \real{0.0600}}@{}}
\toprule\noalign{}
\begin{minipage}[b]{\linewidth}\raggedright
skim\_variable
\end{minipage} & \begin{minipage}[b]{\linewidth}\raggedleft
n\_missing
\end{minipage} & \begin{minipage}[b]{\linewidth}\raggedleft
complete\_rate
\end{minipage} & \begin{minipage}[b]{\linewidth}\raggedleft
mean
\end{minipage} & \begin{minipage}[b]{\linewidth}\raggedleft
sd
\end{minipage} & \begin{minipage}[b]{\linewidth}\raggedleft
p0
\end{minipage} & \begin{minipage}[b]{\linewidth}\raggedleft
p25
\end{minipage} & \begin{minipage}[b]{\linewidth}\raggedleft
p50
\end{minipage} & \begin{minipage}[b]{\linewidth}\raggedleft
p75
\end{minipage} & \begin{minipage}[b]{\linewidth}\raggedleft
p100
\end{minipage} & \begin{minipage}[b]{\linewidth}\raggedright
hist
\end{minipage} \\
\midrule\noalign{}
\endhead
\bottomrule\noalign{}
\endlastfoot
FreelancerID & 0 & 1 & 975.50 & 563.06 & 1.00 & 488.25 & 975.50 &
1462.75 & 1950.00 & ▇▇▇▇▇ \\
JobsCompleted & 0 & 1 & 150.86 & 85.48 & 5.00 & 76.00 & 149.00 & 225.00
& 299.00 & ▇▇▇▇▇ \\
EarningsUSD & 0 & 1 & 5017.57 & 2926.28 & 51.00 & 2419.00 & 5048.00 &
7608.25 & 9991.00 & ▇▇▇▇▇ \\
HourlyRate & 0 & 1 & 52.58 & 26.93 & 5.02 & 30.05 & 52.28 & 75.12 &
99.83 & ▇▇▇▇▇ \\
JobSuccessRate & 0 & 1 & 74.95 & 14.62 & 50.16 & 61.92 & 75.40 & 87.54 &
99.99 & ▇▇▇▇▇ \\
ClientRating & 0 & 1 & 4.00 & 0.58 & 3.00 & 3.51 & 3.99 & 4.50 & 5.00 &
▇▇▇▇▇ \\
JobDurationDays & 0 & 1 & 44.70 & 26.02 & 1.00 & 22.00 & 45.00 & 67.00 &
89.00 & ▇▇▇▇▇ \\
RehireRate & 0 & 1 & 44.56 & 20.19 & 10.00 & 27.15 & 43.92 & 61.69 &
79.95 & ▇▇▇▇▇ \\
MarketingSpend & 0 & 1 & 248.52 & 148.08 & 0.00 & 119.00 & 252.50 &
379.00 & 499.00 & ▇▇▇▇▇ \\
\end{longtable}

\textbf{Phân tích:}

\begin{itemize}
\item
  Tập dữ liệu có r nrow(data\_clean) quan sát và r ncol(data\_clean)
  biến.
\item
  \textbf{EarningsUSD}: Thu nhập trung bình là r
  scales::dollar(round(mean(data\_clean\$EarningsUSD),0)), với độ lệch
  chuẩn khá lớn, cho thấy sự biến động cao trong thu nhập. Giá trị trung
  vị là r scales::dollar(median(data\_clean\$EarningsUSD)). Khoảng thu
  nhập rộng từ r scales::dollar(min(data\_clean\$EarningsUSD)) đến r
  scales::dollar(max(data\_clean\$EarningsUSD)).
\item
  \textbf{HourlyRate}: Mức lương theo giờ trung bình là r
  scales::dollar(round(mean(data\_clean\$HourlyRate),2)).
\item
  \textbf{JobSuccessRate}: Tỷ lệ thành công trung bình là r
  round(mean(data\_clean\$JobSuccessRate),2)\%.
\item
  \textbf{ClientRating}: Đánh giá khách hàng trung bình là r
  round(mean(data\_clean\$ClientRating),2).
\item
  Các biến categorical như JobCategory, Platform, ExperienceLevel có số
  lượng các mức khác nhau, sẽ được khám phá chi tiết hơn bằng biểu đồ.
\end{itemize}

\subsection{3.2. Phân tích Đơn biến (Univariate
Analysis)}\label{phuxe2n-tuxedch-ux111ux1a1n-biux1ebfn-univariate-analysis}

\subsubsection{3.2.1. Phân phối Thu nhập
(EarningsUSD)}\label{phuxe2n-phux1ed1i-thu-nhux1eadp-earningsusd}

\begin{Shaded}
\begin{Highlighting}[]
\FunctionTok{ggplot}\NormalTok{(data\_clean, }\FunctionTok{aes}\NormalTok{(}\AttributeTok{x =}\NormalTok{ EarningsUSD)) }\SpecialCharTok{+}
  \FunctionTok{geom\_histogram}\NormalTok{(}\AttributeTok{binwidth =} \DecValTok{500}\NormalTok{, }\AttributeTok{fill =} \StringTok{"steelblue"}\NormalTok{, }\AttributeTok{color =} \StringTok{"white"}\NormalTok{, }\AttributeTok{alpha =} \FloatTok{0.8}\NormalTok{) }\SpecialCharTok{+}
  \FunctionTok{labs}\NormalTok{(}\AttributeTok{title =} \StringTok{"Phân phối Thu nhập của Freelancer (EarningsUSD)"}\NormalTok{,}
       \AttributeTok{x =} \StringTok{"Thu nhập (USD)"}\NormalTok{,}
       \AttributeTok{y =} \StringTok{"Số lượng Freelancer"}\NormalTok{) }\SpecialCharTok{+}
  \FunctionTok{theme\_minimal}\NormalTok{() }\SpecialCharTok{+}
  \FunctionTok{scale\_x\_continuous}\NormalTok{(}\AttributeTok{labels =}\NormalTok{ scales}\SpecialCharTok{::}\NormalTok{comma)}
\end{Highlighting}
\end{Shaded}

\begin{center}\includegraphics{api_files/figure-latex/earnings-distribution-1} \end{center}

\begin{Shaded}
\begin{Highlighting}[]
\FunctionTok{ggplot}\NormalTok{(data\_clean, }\FunctionTok{aes}\NormalTok{(}\AttributeTok{x =} \StringTok{""}\NormalTok{, }\AttributeTok{y =}\NormalTok{ EarningsUSD)) }\SpecialCharTok{+}
  \FunctionTok{geom\_boxplot}\NormalTok{(}\AttributeTok{fill =} \StringTok{"skyblue"}\NormalTok{, }\AttributeTok{color =} \StringTok{"black"}\NormalTok{, }\AttributeTok{outlier.colour =} \StringTok{"red"}\NormalTok{, }\AttributeTok{outlier.shape =} \DecValTok{1}\NormalTok{) }\SpecialCharTok{+}
  \FunctionTok{labs}\NormalTok{(}\AttributeTok{title =} \StringTok{"Boxplot Phân phối Thu nhập (EarningsUSD)"}\NormalTok{,}
       \AttributeTok{y =} \StringTok{"Thu nhập (USD)"}\NormalTok{) }\SpecialCharTok{+}
  \FunctionTok{theme\_minimal}\NormalTok{() }\SpecialCharTok{+} 
  \FunctionTok{coord\_flip}\NormalTok{() }\SpecialCharTok{+} \CommentTok{\# Xoay boxplot cho dễ nhìn hơn khi chỉ có 1 nhóm}
  \FunctionTok{scale\_y\_continuous}\NormalTok{(}\AttributeTok{labels =}\NormalTok{ scales}\SpecialCharTok{::}\NormalTok{comma)}
\end{Highlighting}
\end{Shaded}

\begin{center}\includegraphics{api_files/figure-latex/earnings-distribution-2} \end{center}

\textbf{Ý nghĩa và Phân tích:}

\begin{itemize}
\item
  \textbf{Histogram}: Biểu đồ histogram cho thấy phân phối của
  EarningsUSD bị lệch phải (right-skewed), nghĩa là phần lớn freelancer
  có thu nhập ở mức thấp đến trung bình, và một số ít freelancer có thu
  nhập rất cao.
\item
  \textbf{Boxplot}: Boxplot cũng xác nhận điều này, với nhiều điểm ngoại
  lệ (outliers) ở phía thu nhập cao. Khoảng tứ phân vị (IQR) cho thấy sự
  tập trung của 50\% dữ liệu ở giữa.
\item
  \textbf{Tri thức mới}: Sự chênh lệch lớn về thu nhập cho thấy có thể
  có các nhóm freelancer khác nhau hoặc các yếu tố đặc biệt dẫn đến thu
  nhập cao vượt trội.
\end{itemize}

\subsubsection{3.2.2. Phân phối Tỷ lệ Giờ làm việc
(HourlyRate)}\label{phuxe2n-phux1ed1i-tux1ef7-lux1ec7-giux1edd-luxe0m-viux1ec7c-hourlyrate}

\begin{Shaded}
\begin{Highlighting}[]
\FunctionTok{ggplot}\NormalTok{(data\_clean, }\FunctionTok{aes}\NormalTok{(}\AttributeTok{x =}\NormalTok{ HourlyRate)) }\SpecialCharTok{+}
  \FunctionTok{geom\_histogram}\NormalTok{(}\AttributeTok{binwidth =} \DecValTok{5}\NormalTok{, }\AttributeTok{fill =} \StringTok{"coral"}\NormalTok{, }\AttributeTok{color =} \StringTok{"white"}\NormalTok{, }\AttributeTok{alpha =} \FloatTok{0.8}\NormalTok{) }\SpecialCharTok{+}
  \FunctionTok{labs}\NormalTok{(}\AttributeTok{title =} \StringTok{"Phân phối Mức lương theo giờ (HourlyRate)"}\NormalTok{,}
       \AttributeTok{x =} \StringTok{"Mức lương theo giờ (USD)"}\NormalTok{,}
       \AttributeTok{y =} \StringTok{"Số lượng Freelancer"}\NormalTok{) }\SpecialCharTok{+}
  \FunctionTok{theme\_minimal}\NormalTok{() }\SpecialCharTok{+}
  \FunctionTok{scale\_x\_continuous}\NormalTok{(}\AttributeTok{labels =}\NormalTok{ scales}\SpecialCharTok{::}\NormalTok{comma)}
\end{Highlighting}
\end{Shaded}

\begin{center}\includegraphics{api_files/figure-latex/hourly-rate-distribution-1} \end{center}

\textbf{Ý nghĩa và Phân tích:}

\begin{itemize}
\tightlist
\item
  Phân phối của HourlyRate cũng có xu hướng lệch phải, nhưng không rõ
  rệt bằng EarningsUSD. Phần lớn freelancer có mức lương theo giờ tập
  trung ở khoảng thấp và trung bình. Có một số freelancer đạt mức lương
  giờ rất cao.
\end{itemize}

\subsubsection{3.2.3. Phân loại Công việc
(JobCategory)}\label{phuxe2n-loux1ea1i-cuxf4ng-viux1ec7c-jobcategory}

\begin{Shaded}
\begin{Highlighting}[]
\NormalTok{data\_clean }\SpecialCharTok{\%\textgreater{}\%}
  \FunctionTok{count}\NormalTok{(JobCategory, }\AttributeTok{sort =} \ConstantTok{TRUE}\NormalTok{) }\SpecialCharTok{\%\textgreater{}\%}
  \FunctionTok{ggplot}\NormalTok{(}\FunctionTok{aes}\NormalTok{(}\AttributeTok{x =} \FunctionTok{reorder}\NormalTok{(JobCategory, }\SpecialCharTok{{-}}\NormalTok{n), }\AttributeTok{y =}\NormalTok{ n)) }\SpecialCharTok{+}
  \FunctionTok{geom\_bar}\NormalTok{(}\AttributeTok{stat =} \StringTok{"identity"}\NormalTok{, }\AttributeTok{fill =} \StringTok{"lightgreen"}\NormalTok{, }\AttributeTok{color =} \StringTok{"black"}\NormalTok{, }\AttributeTok{alpha =} \FloatTok{0.8}\NormalTok{) }\SpecialCharTok{+}
  \FunctionTok{geom\_text}\NormalTok{(}\FunctionTok{aes}\NormalTok{(}\AttributeTok{label =}\NormalTok{ n), }\AttributeTok{vjust =} \SpecialCharTok{{-}}\FloatTok{0.5}\NormalTok{, }\AttributeTok{size =} \DecValTok{3}\NormalTok{) }\SpecialCharTok{+}
  \FunctionTok{labs}\NormalTok{(}\AttributeTok{title =} \StringTok{"Số lượng Freelancer theo Loại Công việc"}\NormalTok{,}
       \AttributeTok{x =} \StringTok{"Loại Công việc"}\NormalTok{,}
       \AttributeTok{y =} \StringTok{"Số lượng Freelancer"}\NormalTok{) }\SpecialCharTok{+}
  \FunctionTok{theme\_minimal}\NormalTok{() }\SpecialCharTok{+}
  \FunctionTok{theme}\NormalTok{(}\AttributeTok{axis.text.x =} \FunctionTok{element\_text}\NormalTok{(}\AttributeTok{angle =} \DecValTok{45}\NormalTok{, }\AttributeTok{hjust =} \DecValTok{1}\NormalTok{))}
\end{Highlighting}
\end{Shaded}

\begin{center}\includegraphics{api_files/figure-latex/job-category-distribution-1} \end{center}

\textbf{Ý nghĩa và Phân tích:}

\begin{itemize}
\item
  Biểu đồ cột cho thấy các loại công việc phổ biến nhất. ``Data Entry'',
  ``Web Development'', ``Graphic Design'', và ``Customer Support'' là
  những lĩnh vực có nhiều freelancer tham gia nhất trong tập dữ liệu
  này.
\item
  \textbf{Tri thức mới}: Điều này cho thấy nhu cầu hoặc sự phổ biến của
  các loại công việc này trên các nền tảng freelancer.
\end{itemize}

\subsubsection{3.2.4. Nền tảng Làm việc
(Platform)}\label{nux1ec1n-tux1ea3ng-luxe0m-viux1ec7c-platform}

\begin{Shaded}
\begin{Highlighting}[]
\NormalTok{data\_clean }\SpecialCharTok{\%\textgreater{}\%}
  \FunctionTok{count}\NormalTok{(Platform, }\AttributeTok{sort =} \ConstantTok{TRUE}\NormalTok{) }\SpecialCharTok{\%\textgreater{}\%}
  \FunctionTok{ggplot}\NormalTok{(}\FunctionTok{aes}\NormalTok{(}\AttributeTok{x =} \FunctionTok{reorder}\NormalTok{(Platform, }\SpecialCharTok{{-}}\NormalTok{n), }\AttributeTok{y =}\NormalTok{ n)) }\SpecialCharTok{+}
  \FunctionTok{geom\_bar}\NormalTok{(}\AttributeTok{stat =} \StringTok{"identity"}\NormalTok{, }\AttributeTok{fill =} \StringTok{"gold"}\NormalTok{, }\AttributeTok{color =} \StringTok{"black"}\NormalTok{, }\AttributeTok{alpha =} \FloatTok{0.8}\NormalTok{) }\SpecialCharTok{+}
  \FunctionTok{geom\_text}\NormalTok{(}\FunctionTok{aes}\NormalTok{(}\AttributeTok{label =}\NormalTok{ n), }\AttributeTok{vjust =} \SpecialCharTok{{-}}\FloatTok{0.5}\NormalTok{, }\AttributeTok{size =} \DecValTok{3}\NormalTok{) }\SpecialCharTok{+}
  \FunctionTok{labs}\NormalTok{(}\AttributeTok{title =} \StringTok{"Số lượng Freelancer theo Nền tảng"}\NormalTok{,}
       \AttributeTok{x =} \StringTok{"Nền tảng"}\NormalTok{,}
       \AttributeTok{y =} \StringTok{"Số lượng Freelancer"}\NormalTok{) }\SpecialCharTok{+}
  \FunctionTok{theme\_minimal}\NormalTok{() }\SpecialCharTok{+}
  \FunctionTok{theme}\NormalTok{(}\AttributeTok{axis.text.x =} \FunctionTok{element\_text}\NormalTok{(}\AttributeTok{angle =} \DecValTok{45}\NormalTok{, }\AttributeTok{hjust =} \DecValTok{1}\NormalTok{))}
\end{Highlighting}
\end{Shaded}

\begin{center}\includegraphics{api_files/figure-latex/platform-distribution-1} \end{center}

\textbf{Ý nghĩa và Phân tích:}

\begin{itemize}
\item
  ``Fiverr'', ``Upwork'', và ``Toptal'' là các nền tảng có nhiều
  freelancer nhất trong mẫu dữ liệu này.
\item
  \textbf{Tri thức mới}: Sự phân bổ này có thể phản ánh mức độ phổ biến
  của các nền tảng hoặc đặc điểm của tập dữ liệu được thu thập.
\end{itemize}

\subsubsection{3.2.5. Mức độ Kinh nghiệm
(ExperienceLevel)}\label{mux1ee9c-ux111ux1ed9-kinh-nghiux1ec7m-experiencelevel}

\begin{Shaded}
\begin{Highlighting}[]
\NormalTok{data\_clean }\SpecialCharTok{\%\textgreater{}\%}
  \FunctionTok{count}\NormalTok{(ExperienceLevel, }\AttributeTok{sort =} \ConstantTok{FALSE}\NormalTok{) }\SpecialCharTok{\%\textgreater{}\%} \CommentTok{\# Giữ nguyên thứ tự của factor đã định nghĩa}
  \FunctionTok{ggplot}\NormalTok{(}\FunctionTok{aes}\NormalTok{(}\AttributeTok{x =}\NormalTok{ ExperienceLevel, }\AttributeTok{y =}\NormalTok{ n)) }\SpecialCharTok{+} 
  \FunctionTok{geom\_bar}\NormalTok{(}\AttributeTok{stat =} \StringTok{"identity"}\NormalTok{, }\AttributeTok{fill =} \StringTok{"orchid"}\NormalTok{, }\AttributeTok{color =} \StringTok{"black"}\NormalTok{, }\AttributeTok{alpha =} \FloatTok{0.8}\NormalTok{) }\SpecialCharTok{+}
  \FunctionTok{geom\_text}\NormalTok{(}\FunctionTok{aes}\NormalTok{(}\AttributeTok{label =}\NormalTok{ n), }\AttributeTok{vjust =} \SpecialCharTok{{-}}\FloatTok{0.5}\NormalTok{, }\AttributeTok{size =} \DecValTok{3}\NormalTok{) }\SpecialCharTok{+}
  \FunctionTok{labs}\NormalTok{(}\AttributeTok{title =} \StringTok{"Số lượng Freelancer theo Mức độ Kinh nghiệm"}\NormalTok{,}
       \AttributeTok{x =} \StringTok{"Mức độ Kinh nghiệm"}\NormalTok{,}
       \AttributeTok{y =} \StringTok{"Số lượng Freelancer"}\NormalTok{) }\SpecialCharTok{+}
  \FunctionTok{theme\_minimal}\NormalTok{()}
\end{Highlighting}
\end{Shaded}

\begin{center}\includegraphics{api_files/figure-latex/experience-level-distribution-1} \end{center}

\textbf{Ý nghĩa và Phân tích:}

\begin{itemize}
\item
  Số lượng freelancer ở các mức kinh nghiệm ``Beginner'',
  ``Intermediate'', và ``Expert'' có sự phân bổ khá đều, với
  ``Beginner'' và ``Intermediate'' chiếm tỷ lệ cao hơn một chút.
\item
  \textbf{Tri thức mới}: Điều này cho thấy thị trường freelancer có sự
  tham gia của nhiều người ở các giai đoạn sự nghiệp khác nhau.
\end{itemize}

\subsubsection{3.2.6. Loại Hình Dự Án
(ProjectType)}\label{loux1ea1i-huxecnh-dux1ef1-uxe1n-projecttype}

\begin{Shaded}
\begin{Highlighting}[]
\NormalTok{project\_type\_summary }\OtherTok{\textless{}{-}}\NormalTok{ data\_clean }\SpecialCharTok{\%\textgreater{}\%}
  \FunctionTok{group\_by}\NormalTok{(ProjectType) }\SpecialCharTok{\%\textgreater{}\%}
  \FunctionTok{summarise}\NormalTok{(}\AttributeTok{count =} \FunctionTok{n}\NormalTok{(),}
            \AttributeTok{percentage =} \FunctionTok{n}\NormalTok{() }\SpecialCharTok{/} \FunctionTok{nrow}\NormalTok{(data\_clean) }\SpecialCharTok{*} \DecValTok{100}\NormalTok{) }\SpecialCharTok{\%\textgreater{}\%}
  \FunctionTok{ungroup}\NormalTok{()}

\FunctionTok{ggplot}\NormalTok{(project\_type\_summary, }\FunctionTok{aes}\NormalTok{(}\AttributeTok{x =}\NormalTok{ ProjectType, }\AttributeTok{y =}\NormalTok{ count, }\AttributeTok{fill =}\NormalTok{ ProjectType)) }\SpecialCharTok{+}
  \FunctionTok{geom\_bar}\NormalTok{(}\AttributeTok{stat =} \StringTok{"identity"}\NormalTok{, }\AttributeTok{color =} \StringTok{"black"}\NormalTok{, }\AttributeTok{alpha =} \FloatTok{0.8}\NormalTok{) }\SpecialCharTok{+}
  \FunctionTok{geom\_text}\NormalTok{(}\FunctionTok{aes}\NormalTok{(}\AttributeTok{label =} \FunctionTok{paste0}\NormalTok{(}\FunctionTok{round}\NormalTok{(percentage,}\DecValTok{1}\NormalTok{), }\StringTok{"\% ("}\NormalTok{, count, }\StringTok{")"}\NormalTok{)), }\AttributeTok{vjust =} \SpecialCharTok{{-}}\FloatTok{0.5}\NormalTok{, }\AttributeTok{size =} \FloatTok{3.5}\NormalTok{) }\SpecialCharTok{+}
  \FunctionTok{labs}\NormalTok{(}\AttributeTok{title =} \StringTok{"Phân bổ Freelancer theo Loại Hình Dự Án"}\NormalTok{,}
       \AttributeTok{x =} \StringTok{"Loại Hình Dự Án"}\NormalTok{,}
       \AttributeTok{y =} \StringTok{"Số lượng Freelancer"}\NormalTok{) }\SpecialCharTok{+}
  \FunctionTok{theme\_minimal}\NormalTok{() }\SpecialCharTok{+}
  \FunctionTok{scale\_fill\_brewer}\NormalTok{(}\AttributeTok{palette =} \StringTok{"Pastel1"}\NormalTok{)}
\end{Highlighting}
\end{Shaded}

\begin{center}\includegraphics{api_files/figure-latex/project-type-distribution-1} \end{center}

\textbf{Ý nghĩa và Phân tích:}

\begin{itemize}
\item
  Biểu đồ cho thấy sự phân bổ giữa các dự án tính theo giờ (Hourly) và
  dự án giá cố định (Fixed). Trong tập dữ liệu này, số lượng dự án tính
  theo giờ và giá cố định khá tương đồng, với dự án theo giờ chiếm tỷ lệ
  nhỉnh hơn một chút.
\item
  \textbf{Tri thức mới}: Cả hai mô hình định giá đều phổ biến,
  freelancer có thể lựa chọn tùy theo tính chất công việc và sở thích.
\end{itemize}

\subsection{3.3. Phân tích Đa biến (Bivariate \& Multivariate
Analysis)}\label{phuxe2n-tuxedch-ux111a-biux1ebfn-bivariate-multivariate-analysis}

\subsubsection{3.3.1. Thu nhập theo Loại Công
việc}\label{thu-nhux1eadp-theo-loux1ea1i-cuxf4ng-viux1ec7c}

\begin{Shaded}
\begin{Highlighting}[]
\FunctionTok{ggplot}\NormalTok{(data\_clean, }\FunctionTok{aes}\NormalTok{(}\AttributeTok{x =} \FunctionTok{reorder}\NormalTok{(JobCategory, EarningsUSD, }\AttributeTok{FUN =}\NormalTok{ median, }\AttributeTok{.desc =} \ConstantTok{TRUE}\NormalTok{), }\AttributeTok{y =}\NormalTok{ EarningsUSD, }\AttributeTok{fill =}\NormalTok{ JobCategory)) }\SpecialCharTok{+}
  \FunctionTok{geom\_boxplot}\NormalTok{(}\AttributeTok{alpha =} \FloatTok{0.8}\NormalTok{, }\AttributeTok{outlier.shape =} \DecValTok{1}\NormalTok{) }\SpecialCharTok{+}
  \FunctionTok{labs}\NormalTok{(}\AttributeTok{title =} \StringTok{"Phân phối Thu nhập (EarningsUSD) theo Loại Công việc"}\NormalTok{,}
       \AttributeTok{x =} \StringTok{"Loại Công việc"}\NormalTok{,}
       \AttributeTok{y =} \StringTok{"Thu nhập (USD)"}\NormalTok{) }\SpecialCharTok{+}
  \FunctionTok{theme\_minimal}\NormalTok{() }\SpecialCharTok{+}
  \FunctionTok{theme}\NormalTok{(}\AttributeTok{axis.text.x =} \FunctionTok{element\_text}\NormalTok{(}\AttributeTok{angle =} \DecValTok{45}\NormalTok{, }\AttributeTok{hjust =} \DecValTok{1}\NormalTok{), }\AttributeTok{legend.position =} \StringTok{"none"}\NormalTok{) }\SpecialCharTok{+}
  \FunctionTok{scale\_y\_log10}\NormalTok{(}\AttributeTok{labels =}\NormalTok{ scales}\SpecialCharTok{::}\NormalTok{comma) }\CommentTok{\# Sử dụng thang log cho y để dễ nhìn hơn do độ lệch lớn}
\end{Highlighting}
\end{Shaded}

\begin{center}\includegraphics{api_files/figure-latex/earnings-by-job-category-1} \end{center}

\textbf{Ý nghĩa và Phân tích:}

\begin{itemize}
\item
  Biểu đồ boxplot cho thấy sự khác biệt đáng kể về thu nhập trung vị
  giữa các loại công việc.
\item
  ``App Development'' và ``Web Development'' dường như có mức thu nhập
  trung vị cao hơn so với các ngành khác như ``Data Entry'' hay
  ``Customer Support''.
\item
  Sự biến động (độ dài của box và whiskers) cũng khác nhau giữa các
  ngành.
\item
  \textbf{Tri thức mới}: Lựa chọn ngành nghề có ảnh hưởng lớn đến tiềm
  năng thu nhập. Các ngành kỹ thuật cao thường có thu nhập tốt hơn.
\end{itemize}

\subsubsection{3.3.2. Thu nhập theo Nền
tảng}\label{thu-nhux1eadp-theo-nux1ec1n-tux1ea3ng}

\begin{Shaded}
\begin{Highlighting}[]
\FunctionTok{ggplot}\NormalTok{(data\_clean, }\FunctionTok{aes}\NormalTok{(}\AttributeTok{x =} \FunctionTok{reorder}\NormalTok{(Platform, EarningsUSD, }\AttributeTok{FUN =}\NormalTok{ median, }\AttributeTok{.desc =} \ConstantTok{TRUE}\NormalTok{), }\AttributeTok{y =}\NormalTok{ EarningsUSD, }\AttributeTok{fill =}\NormalTok{ Platform)) }\SpecialCharTok{+}
  \FunctionTok{geom\_boxplot}\NormalTok{(}\AttributeTok{alpha =} \FloatTok{0.8}\NormalTok{, }\AttributeTok{outlier.shape =} \DecValTok{1}\NormalTok{) }\SpecialCharTok{+}
  \FunctionTok{labs}\NormalTok{(}\AttributeTok{title =} \StringTok{"Phân phối Thu nhập (EarningsUSD) theo Nền tảng"}\NormalTok{,}
       \AttributeTok{x =} \StringTok{"Nền tảng"}\NormalTok{,}
       \AttributeTok{y =} \StringTok{"Thu nhập (USD)"}\NormalTok{) }\SpecialCharTok{+}
  \FunctionTok{theme\_minimal}\NormalTok{() }\SpecialCharTok{+}
  \FunctionTok{theme}\NormalTok{(}\AttributeTok{axis.text.x =} \FunctionTok{element\_text}\NormalTok{(}\AttributeTok{angle =} \DecValTok{45}\NormalTok{, }\AttributeTok{hjust =} \DecValTok{1}\NormalTok{), }\AttributeTok{legend.position =} \StringTok{"none"}\NormalTok{) }\SpecialCharTok{+}
  \FunctionTok{scale\_y\_log10}\NormalTok{(}\AttributeTok{labels =}\NormalTok{ scales}\SpecialCharTok{::}\NormalTok{comma)}
\end{Highlighting}
\end{Shaded}

\begin{center}\includegraphics{api_files/figure-latex/earnings-by-platform-1} \end{center}

\textbf{Ý nghĩa và Phân tích:}

\begin{itemize}
\item
  Có sự khác biệt về thu nhập trung vị giữa các nền tảng. Ví dụ,
  ``Toptal'' thường được biết đến với các freelancer chất lượng cao và
  dự án giá trị lớn, có thể dẫn đến thu nhập trung bình cao hơn.
\item
  \textbf{Tri thức mới}: Nền tảng làm việc cũng là một yếu tố quan
  trọng. Một số nền tảng có thể tập trung vào các phân khúc thị trường
  hoặc loại dự án khác nhau, ảnh hưởng đến thu nhập.
\end{itemize}

\subsubsection{3.3.3. Thu nhập theo Mức độ Kinh
nghiệm}\label{thu-nhux1eadp-theo-mux1ee9c-ux111ux1ed9-kinh-nghiux1ec7m}

\begin{Shaded}
\begin{Highlighting}[]
\FunctionTok{ggplot}\NormalTok{(data\_clean, }\FunctionTok{aes}\NormalTok{(}\AttributeTok{x =}\NormalTok{ ExperienceLevel, }\AttributeTok{y =}\NormalTok{ EarningsUSD, }\AttributeTok{fill =}\NormalTok{ ExperienceLevel)) }\SpecialCharTok{+}
  \FunctionTok{geom\_boxplot}\NormalTok{(}\AttributeTok{alpha =} \FloatTok{0.8}\NormalTok{, }\AttributeTok{outlier.shape =} \DecValTok{1}\NormalTok{) }\SpecialCharTok{+}
  \FunctionTok{labs}\NormalTok{(}\AttributeTok{title =} \StringTok{"Phân phối Thu nhập (EarningsUSD) theo Mức độ Kinh nghiệm"}\NormalTok{,}
       \AttributeTok{x =} \StringTok{"Mức độ Kinh nghiệm"}\NormalTok{,}
       \AttributeTok{y =} \StringTok{"Thu nhập (USD)"}\NormalTok{) }\SpecialCharTok{+}
  \FunctionTok{theme\_minimal}\NormalTok{() }\SpecialCharTok{+}
  \FunctionTok{theme}\NormalTok{(}\AttributeTok{legend.position =} \StringTok{"none"}\NormalTok{) }\SpecialCharTok{+}
  \FunctionTok{scale\_y\_log10}\NormalTok{(}\AttributeTok{labels =}\NormalTok{ scales}\SpecialCharTok{::}\NormalTok{comma) }
\end{Highlighting}
\end{Shaded}

\begin{center}\includegraphics{api_files/figure-latex/earnings-by-experience-level-1} \end{center}

\textbf{Ý nghĩa và Phân tích:}

\begin{itemize}
\item
  Rõ ràng có xu hướng thu nhập tăng theo mức độ kinh nghiệm. Freelancer
  ``Expert'' có thu nhập trung vị cao nhất, tiếp theo là
  ``Intermediate'' và cuối cùng là ``Beginner''.
\item
  \textbf{Tri thức mới}: Đầu tư vào việc nâng cao kỹ năng và kinh nghiệm
  là một chiến lược quan trọng để tăng thu nhập.
\end{itemize}

\subsubsection{3.3.4. Mối quan hệ giữa Số lượng Công việc Hoàn thành và
Thu
nhập}\label{mux1ed1i-quan-hux1ec7-giux1eefa-sux1ed1-lux1b0ux1ee3ng-cuxf4ng-viux1ec7c-houxe0n-thuxe0nh-vuxe0-thu-nhux1eadp}

\begin{Shaded}
\begin{Highlighting}[]
\FunctionTok{ggplot}\NormalTok{(data\_clean, }\FunctionTok{aes}\NormalTok{(}\AttributeTok{x =}\NormalTok{ JobsCompleted, }\AttributeTok{y =}\NormalTok{ EarningsUSD)) }\SpecialCharTok{+}
  \FunctionTok{geom\_point}\NormalTok{(}\FunctionTok{aes}\NormalTok{(}\AttributeTok{color =}\NormalTok{ ExperienceLevel), }\AttributeTok{alpha =} \FloatTok{0.5}\NormalTok{) }\SpecialCharTok{+}
  \FunctionTok{geom\_smooth}\NormalTok{(}\AttributeTok{method =} \StringTok{"lm"}\NormalTok{, }\AttributeTok{se =} \ConstantTok{FALSE}\NormalTok{, }\AttributeTok{color =} \StringTok{"darkred"}\NormalTok{) }\SpecialCharTok{+} \CommentTok{\# Thêm đường xu hướng tuyến tính}
  \FunctionTok{labs}\NormalTok{(}\AttributeTok{title =} \StringTok{"Mối quan hệ giữa Số lượng Công việc Hoàn thành và Thu nhập"}\NormalTok{,}
       \AttributeTok{x =} \StringTok{"Số lượng Công việc Hoàn thành"}\NormalTok{,}
       \AttributeTok{y =} \StringTok{"Thu nhập (USD)"}\NormalTok{) }\SpecialCharTok{+}
  \FunctionTok{theme\_minimal}\NormalTok{() }\SpecialCharTok{+}
  \FunctionTok{scale\_y\_log10}\NormalTok{(}\AttributeTok{labels =}\NormalTok{ scales}\SpecialCharTok{::}\NormalTok{comma) }\SpecialCharTok{+}
  \FunctionTok{scale\_x\_log10}\NormalTok{(}\AttributeTok{labels =}\NormalTok{ scales}\SpecialCharTok{::}\NormalTok{comma) }\CommentTok{\# Thang log cho cả 2 trục nếu cần}
\end{Highlighting}
\end{Shaded}

\begin{center}\includegraphics{api_files/figure-latex/earnings-vs-jobs-completed-1} \end{center}

\textbf{Ý nghĩa và Phân tích:}

\begin{itemize}
\item
  Biểu đồ scatter plot cho thấy có một mối quan hệ dương giữa số lượng
  công việc hoàn thành (JobsCompleted) và thu nhập (EarningsUSD). Tuy
  nhiên, mối quan hệ này có thể không hoàn toàn tuyến tính và có nhiều
  biến động.
\item
  Các freelancer có kinh nghiệm (Expert) thường có thể hoàn thành nhiều
  việc hơn và có thu nhập cao hơn, nhưng cũng có những người mới bắt đầu
  (Beginner) hoàn thành nhiều việc với thu nhập thấp hơn.
\item
  \textbf{Tri thức mới}: Hoàn thành nhiều công việc thường dẫn đến thu
  nhập cao hơn, nhưng chất lượng và giá trị của mỗi công việc cũng quan
  trọng.
\end{itemize}

\subsubsection{3.3.5. Mối quan hệ giữa Tỷ lệ Thành công và Thu
nhập}\label{mux1ed1i-quan-hux1ec7-giux1eefa-tux1ef7-lux1ec7-thuxe0nh-cuxf4ng-vuxe0-thu-nhux1eadp}

\begin{Shaded}
\begin{Highlighting}[]
\FunctionTok{ggplot}\NormalTok{(data\_clean, }\FunctionTok{aes}\NormalTok{(}\AttributeTok{x =}\NormalTok{ JobSuccessRate, }\AttributeTok{y =}\NormalTok{ EarningsUSD)) }\SpecialCharTok{+}
  \FunctionTok{geom\_point}\NormalTok{(}\FunctionTok{aes}\NormalTok{(}\AttributeTok{color =}\NormalTok{ ExperienceLevel), }\AttributeTok{alpha =} \FloatTok{0.5}\NormalTok{) }\SpecialCharTok{+}
  \FunctionTok{geom\_smooth}\NormalTok{(}\AttributeTok{method =} \StringTok{"loess"}\NormalTok{, }\AttributeTok{se =} \ConstantTok{FALSE}\NormalTok{, }\AttributeTok{color =} \StringTok{"darkblue"}\NormalTok{) }\SpecialCharTok{+} \CommentTok{\# Đường xu hướng mượt hơn}
  \FunctionTok{labs}\NormalTok{(}\AttributeTok{title =} \StringTok{"Mối quan hệ giữa Tỷ lệ Thành công và Thu nhập"}\NormalTok{,}
       \AttributeTok{x =} \StringTok{"Tỷ lệ Thành công (\%)"}\NormalTok{,}
       \AttributeTok{y =} \StringTok{"Thu nhập (USD)"}\NormalTok{) }\SpecialCharTok{+}
  \FunctionTok{theme\_minimal}\NormalTok{() }\SpecialCharTok{+}
  \FunctionTok{scale\_y\_log10}\NormalTok{(}\AttributeTok{labels =}\NormalTok{ scales}\SpecialCharTok{::}\NormalTok{comma)}
\end{Highlighting}
\end{Shaded}

\begin{center}\includegraphics{api_files/figure-latex/earnings-vs-job-success-rate-1} \end{center}

\textbf{Ý nghĩa và Phân tích:}

\begin{itemize}
\item
  Dường như có một xu hướng dương nhẹ: tỷ lệ thành công cao hơn có thể
  liên quan đến thu nhập cao hơn, nhưng mối quan hệ không quá mạnh mẽ và
  có nhiều điểm phân tán.
\item
  \textbf{Tri thức mới}: Duy trì tỷ lệ thành công cao là quan trọng,
  nhưng nó không phải là yếu tố duy nhất quyết định thu nhập. Các yếu tố
  khác như loại công việc, giá trị hợp đồng cũng đóng vai trò lớn.
\end{itemize}

\subsubsection{3.3.6. Ma trận Tương quan giữa các Biến
Số}\label{ma-trux1eadn-tux1b0ux1a1ng-quan-giux1eefa-cuxe1c-biux1ebfn-sux1ed1}

\begin{Shaded}
\begin{Highlighting}[]
\CommentTok{\# Chọn các biến số để tính tương quan}
\NormalTok{numeric\_vars }\OtherTok{\textless{}{-}}\NormalTok{ data\_clean }\SpecialCharTok{\%\textgreater{}\%}
  \FunctionTok{select\_if}\NormalTok{(is.numeric) }\SpecialCharTok{\%\textgreater{}\%}
  \FunctionTok{select}\NormalTok{(}\SpecialCharTok{{-}}\NormalTok{FreelancerID) }\CommentTok{\# Loại bỏ ID}

\CommentTok{\# Tính ma trận tương quan}
\NormalTok{cor\_matrix }\OtherTok{\textless{}{-}} \FunctionTok{cor}\NormalTok{(numeric\_vars, }\AttributeTok{use =} \StringTok{"pairwise.complete.obs"}\NormalTok{) }\CommentTok{\# Xử lý NA nếu còn sót}

\CommentTok{\# Vẽ biểu đồ ma trận tương quan}
\FunctionTok{corrplot}\NormalTok{(cor\_matrix, }\AttributeTok{method =} \StringTok{"color"}\NormalTok{, }\AttributeTok{type =} \StringTok{"upper"}\NormalTok{, }\AttributeTok{order =} \StringTok{"hclust"}\NormalTok{,}
         \AttributeTok{addCoef.col =} \StringTok{"black"}\NormalTok{, }\CommentTok{\# Thêm hệ số tương quan}
         \AttributeTok{tl.col =} \StringTok{"black"}\NormalTok{, }\AttributeTok{tl.srt =} \DecValTok{45}\NormalTok{, }\CommentTok{\# Màu và góc của nhãn}
         \AttributeTok{diag =} \ConstantTok{FALSE}\NormalTok{, }\CommentTok{\# Không hiển thị đường chéo}
         \AttributeTok{title =} \StringTok{"Ma trận Tương quan giữa các Biến Số"}\NormalTok{, }\AttributeTok{mar =} \FunctionTok{c}\NormalTok{(}\DecValTok{0}\NormalTok{,}\DecValTok{0}\NormalTok{,}\DecValTok{1}\NormalTok{,}\DecValTok{0}\NormalTok{))}
\end{Highlighting}
\end{Shaded}

\begin{center}\includegraphics{api_files/figure-latex/correlation-matrix-1} \end{center}

\textbf{Ý nghĩa và Phân tích:}

\begin{itemize}
\item
  \textbf{EarningsUSD vs.~JobsCompleted}: Có tương quan dương vừa phải
  (r round(cor\_matrix{[}``EarningsUSD'', ``JobsCompleted''{]}, 2)), cho
  thấy số lượng công việc hoàn thành nhiều hơn thường đi kèm với thu
  nhập cao hơn.
\item
  \textbf{EarningsUSD vs.~HourlyRate}: Mối tương quan này (r
  round(cor\_matrix{[}``EarningsUSD'', ``HourlyRate''{]}, 2)) có thể
  không cao như mong đợi, vì tổng thu nhập còn phụ thuộc vào tổng số giờ
  làm việc hoặc số lượng dự án cố định.
\item
  \textbf{MarketingSpend vs.~EarningsUSD}: Tương quan dương (r
  round(cor\_matrix{[}``EarningsUSD'', ``MarketingSpend''{]}, 2)), cho
  thấy việc chi tiêu cho marketing có thể liên quan đến thu nhập cao
  hơn.
\item
  \textbf{JobSuccessRate vs.~ClientRating}: Thường có tương quan dương
  (r round(cor\_matrix{[}``JobSuccessRate'', ``ClientRating''{]}, 2)),
  tỷ lệ thành công cao thường dẫn đến đánh giá tốt từ khách hàng.
\item
  \textbf{Tri thức mới}: Ma trận tương quan giúp xác định các mối quan
  hệ tuyến tính tiềm năng giữa các biến. Tuy nhiên, cần lưu ý rằng tương
  quan không ngụ ý nhân quả và không phát hiện được các mối quan hệ phi
  tuyến tính.
\end{itemize}

\section{4. Mô hình hóa Dữ
liệu}\label{muxf4-huxecnh-huxf3a-dux1eef-liux1ec7u}

Trong phần này, chúng tôi sẽ xây dựng các mô hình để dự đoán khả năng
một freelancer đạt được ``Thu nhập Cao''.

\subsection{4.1. Chuẩn bị Dữ liệu cho Mô
hình}\label{chuux1ea9n-bux1ecb-dux1eef-liux1ec7u-cho-muxf4-huxecnh}

\subsubsection{4.1.1. Tạo Biến Mục tiêu (Target
Variable)}\label{tux1ea1o-biux1ebfn-mux1ee5c-tiuxeau-target-variable}

Chúng ta sẽ định nghĩa ``Thu nhập Cao'' là những freelancer có thu nhập
(EarningsUSD) nằm trong top 25\% (tức là trên ngưỡng 75th percentile).

\begin{Shaded}
\begin{Highlighting}[]
\CommentTok{\# Xác định ngưỡng thu nhập cao (ví dụ: 75th percentile)}
\NormalTok{threshold\_high\_income }\OtherTok{\textless{}{-}} \FunctionTok{quantile}\NormalTok{(data\_clean}\SpecialCharTok{$}\NormalTok{EarningsUSD, }\FloatTok{0.75}\NormalTok{, }\AttributeTok{na.rm =} \ConstantTok{TRUE}\NormalTok{)}
\FunctionTok{cat}\NormalTok{(}\StringTok{"Ngưỡng thu nhập cao (75th percentile):"}\NormalTok{, scales}\SpecialCharTok{::}\FunctionTok{dollar}\NormalTok{(threshold\_high\_income), }\StringTok{"}\SpecialCharTok{\textbackslash{}n}\StringTok{"}\NormalTok{)}
\end{Highlighting}
\end{Shaded}

\begin{verbatim}
## Ngưỡng thu nhập cao (75th percentile): $7,608.25
\end{verbatim}

\begin{Shaded}
\begin{Highlighting}[]
\NormalTok{data\_model }\OtherTok{\textless{}{-}}\NormalTok{ data\_clean }\SpecialCharTok{\%\textgreater{}\%}
  \FunctionTok{mutate}\NormalTok{(}\AttributeTok{HighEarner =} \FunctionTok{factor}\NormalTok{(}\FunctionTok{ifelse}\NormalTok{(EarningsUSD }\SpecialCharTok{\textgreater{}}\NormalTok{ threshold\_high\_income, }\StringTok{"Yes"}\NormalTok{, }\StringTok{"No"}\NormalTok{), }\AttributeTok{levels =} \FunctionTok{c}\NormalTok{(}\StringTok{"No"}\NormalTok{, }\StringTok{"Yes"}\NormalTok{)))}

\CommentTok{\# Kiểm tra phân bổ của biến mục tiêu}
\FunctionTok{table}\NormalTok{(data\_model}\SpecialCharTok{$}\NormalTok{HighEarner) }\SpecialCharTok{\%\textgreater{}\%} \FunctionTok{kable}\NormalTok{(}\AttributeTok{caption =} \StringTok{"Phân bổ biến mục tiêu HighEarner"}\NormalTok{)}
\end{Highlighting}
\end{Shaded}

\begin{longtable}[]{@{}lr@{}}
\caption{Phân bổ biến mục tiêu HighEarner}\tabularnewline
\toprule\noalign{}
Var1 & Freq \\
\midrule\noalign{}
\endfirsthead
\toprule\noalign{}
Var1 & Freq \\
\midrule\noalign{}
\endhead
\bottomrule\noalign{}
\endlastfoot
No & 1462 \\
Yes & 488 \\
\end{longtable}

\begin{Shaded}
\begin{Highlighting}[]
\FunctionTok{prop.table}\NormalTok{(}\FunctionTok{table}\NormalTok{(data\_model}\SpecialCharTok{$}\NormalTok{HighEarner)) }\SpecialCharTok{\%\textgreater{}\%} \FunctionTok{kable}\NormalTok{(}\AttributeTok{caption =} \StringTok{"Tỷ lệ phân bổ biến mục tiêu HighEarner"}\NormalTok{)}
\end{Highlighting}
\end{Shaded}

\begin{longtable}[]{@{}lr@{}}
\caption{Tỷ lệ phân bổ biến mục tiêu HighEarner}\tabularnewline
\toprule\noalign{}
Var1 & Freq \\
\midrule\noalign{}
\endfirsthead
\toprule\noalign{}
Var1 & Freq \\
\midrule\noalign{}
\endhead
\bottomrule\noalign{}
\endlastfoot
No & 0.7497436 \\
Yes & 0.2502564 \\
\end{longtable}

\begin{Shaded}
\begin{Highlighting}[]
\CommentTok{\# Loại bỏ các biến không cần thiết cho mô hình hoặc biến được dùng để tạo target}
\NormalTok{data\_model }\OtherTok{\textless{}{-}}\NormalTok{ data\_model }\SpecialCharTok{\%\textgreater{}\%} \FunctionTok{select}\NormalTok{(}\SpecialCharTok{{-}}\NormalTok{FreelancerID, }\SpecialCharTok{{-}}\NormalTok{EarningsUSD)}
\end{Highlighting}
\end{Shaded}

\textbf{Nhận xét}: Dữ liệu có sự mất cân bằng nhẹ, với số lượng ``No''
(không phải thu nhập cao) nhiều hơn ``Yes''. Điều này cần được lưu ý khi
đánh giá mô hình.

\subsubsection{4.1.2. Xử lý các biến Categorical và NA (nếu
có)}\label{xux1eed-luxfd-cuxe1c-biux1ebfn-categorical-vuxe0-na-nux1ebfu-cuxf3}

Các mô hình như Logistic Regression yêu cầu các biến đầu vào phải là số.
Chúng ta cần chuyển đổi các biến factor thành dummy variables. Decision
Tree và Random Forest có thể xử lý biến factor trực tiếp. caret sẽ hỗ
trợ việc này.

\begin{Shaded}
\begin{Highlighting}[]
\CommentTok{\# Kiểm tra lại NA trước khi xây dựng mô hình}
\FunctionTok{sapply}\NormalTok{(data\_model, }\ControlFlowTok{function}\NormalTok{(x) }\FunctionTok{sum}\NormalTok{(}\FunctionTok{is.na}\NormalTok{(x))) }\SpecialCharTok{\%\textgreater{}\%} \FunctionTok{kable}\NormalTok{(}\AttributeTok{col.names =} \FunctionTok{c}\NormalTok{(}\StringTok{"Số lượng NA"}\NormalTok{), }\AttributeTok{caption=}\StringTok{"Kiểm tra NA trong data\_model"}\NormalTok{)}
\end{Highlighting}
\end{Shaded}

\begin{longtable}[]{@{}lr@{}}
\caption{Kiểm tra NA trong data\_model}\tabularnewline
\toprule\noalign{}
& Số lượng NA \\
\midrule\noalign{}
\endfirsthead
\toprule\noalign{}
& Số lượng NA \\
\midrule\noalign{}
\endhead
\bottomrule\noalign{}
\endlastfoot
JobCategory & 0 \\
Platform & 0 \\
ExperienceLevel & 0 \\
ClientRegion & 0 \\
PaymentMethod & 0 \\
JobsCompleted & 0 \\
HourlyRate & 0 \\
JobSuccessRate & 0 \\
ClientRating & 0 \\
JobDurationDays & 0 \\
ProjectType & 0 \\
RehireRate & 0 \\
MarketingSpend & 0 \\
HighEarner & 0 \\
\end{longtable}

\begin{Shaded}
\begin{Highlighting}[]
\CommentTok{\# \textasciigrave{}caret\textasciigrave{} sẽ xử lý one{-}hot encoding cho các factor predictors trong quá trình huấn luyện khi cần (vd: cho glm).}
\end{Highlighting}
\end{Shaded}

\subsubsection{4.1.3. Chia Dữ liệu (Train/Test
Split)}\label{chia-dux1eef-liux1ec7u-traintest-split}

Chia dữ liệu thành tập huấn luyện (70\%) và tập kiểm thử (30\%).

\begin{Shaded}
\begin{Highlighting}[]
\FunctionTok{set.seed}\NormalTok{(}\DecValTok{123}\NormalTok{) }\CommentTok{\# Để kết quả có thể tái tạo}
\NormalTok{train\_index }\OtherTok{\textless{}{-}} \FunctionTok{createDataPartition}\NormalTok{(data\_model}\SpecialCharTok{$}\NormalTok{HighEarner, }\AttributeTok{p =} \FloatTok{0.7}\NormalTok{, }\AttributeTok{list =} \ConstantTok{FALSE}\NormalTok{)}
\NormalTok{train\_data }\OtherTok{\textless{}{-}}\NormalTok{ data\_model[train\_index, ]}
\NormalTok{test\_data }\OtherTok{\textless{}{-}}\NormalTok{ data\_model[}\SpecialCharTok{{-}}\NormalTok{train\_index, ]}

\CommentTok{\# Kiểm tra kích thước của các tập dữ liệu}
\FunctionTok{cat}\NormalTok{(}\StringTok{"Số lượng mẫu trong tập huấn luyện:"}\NormalTok{, }\FunctionTok{nrow}\NormalTok{(train\_data), }\StringTok{"}\SpecialCharTok{\textbackslash{}n}\StringTok{"}\NormalTok{)}
\end{Highlighting}
\end{Shaded}

\begin{verbatim}
## Số lượng mẫu trong tập huấn luyện: 1366
\end{verbatim}

\begin{Shaded}
\begin{Highlighting}[]
\FunctionTok{cat}\NormalTok{(}\StringTok{"Số lượng mẫu trong tập kiểm thử:"}\NormalTok{, }\FunctionTok{nrow}\NormalTok{(test\_data), }\StringTok{"}\SpecialCharTok{\textbackslash{}n}\StringTok{"}\NormalTok{)}
\end{Highlighting}
\end{Shaded}

\begin{verbatim}
## Số lượng mẫu trong tập kiểm thử: 584
\end{verbatim}

\subsection{4.2. Xây dựng và Huấn luyện Mô
hình}\label{xuxe2y-dux1ef1ng-vuxe0-huux1ea5n-luyux1ec7n-muxf4-huxecnh}

Chúng ta sẽ sử dụng caret để huấn luyện và đánh giá các mô hình.
trainControl sẽ được sử dụng để thiết lập cross-validation.

\begin{Shaded}
\begin{Highlighting}[]
\CommentTok{\# Thiết lập train control cho cross{-}validation}
\CommentTok{\# Sử dụng 5{-}fold cross{-}validation}
\CommentTok{\# \textquotesingle{}twoClassSummary\textquotesingle{} để tính ROC, Sens, Spec}
\NormalTok{ctrl }\OtherTok{\textless{}{-}} \FunctionTok{trainControl}\NormalTok{(}\AttributeTok{method =} \StringTok{"cv"}\NormalTok{, }
                     \AttributeTok{number =} \DecValTok{5}\NormalTok{, }
                     \AttributeTok{classProbs =} \ConstantTok{TRUE}\NormalTok{, }\CommentTok{\# Cần thiết cho ROC}
                     \AttributeTok{summaryFunction =}\NormalTok{ twoClassSummary, }\CommentTok{\# Để có ROC, Sens, Spec}
                     \AttributeTok{savePredictions =} \StringTok{"final"}\NormalTok{,}
                     \AttributeTok{verboseIter =} \ConstantTok{FALSE}\NormalTok{) }\CommentTok{\# Tắt bớt log khi chạy}
\end{Highlighting}
\end{Shaded}

\subsubsection{4.2.1. Mô hình Hồi quy Logistic (Logistic
Regression)}\label{muxf4-huxecnh-hux1ed3i-quy-logistic-logistic-regression}

\begin{Shaded}
\begin{Highlighting}[]
\CommentTok{\# Huấn luyện mô hình Logistic Regression}
\CommentTok{\# caret sẽ tự động xử lý one{-}hot encoding cho các factor predictors}
\FunctionTok{set.seed}\NormalTok{(}\DecValTok{123}\NormalTok{)}
\NormalTok{logistic\_model }\OtherTok{\textless{}{-}} \FunctionTok{train}\NormalTok{(HighEarner }\SpecialCharTok{\textasciitilde{}}\NormalTok{ ., }
                        \AttributeTok{data =}\NormalTok{ train\_data, }
                        \AttributeTok{method =} \StringTok{"glm"}\NormalTok{, }
                        \AttributeTok{family =} \StringTok{"binomial"}\NormalTok{, }
                        \AttributeTok{trControl =}\NormalTok{ ctrl,}
                        \AttributeTok{metric =} \StringTok{"ROC"}\NormalTok{) }\CommentTok{\# Tối ưu theo ROC}

\CommentTok{\# Xem tóm tắt mô hình}
\FunctionTok{print}\NormalTok{(logistic\_model)}
\end{Highlighting}
\end{Shaded}

\begin{verbatim}
## Generalized Linear Model 
## 
## 1366 samples
##   13 predictor
##    2 classes: 'No', 'Yes' 
## 
## No pre-processing
## Resampling: Cross-Validated (5 fold) 
## Summary of sample sizes: 1093, 1093, 1092, 1093, 1093 
## Resampling results:
## 
##   ROC        Sens       Spec      
##   0.5402652  0.9931612  0.02638534
\end{verbatim}

\begin{Shaded}
\begin{Highlighting}[]
\CommentTok{\# summary(logistic\_model$finalModel) \# Chi tiết các hệ số (có thể rất dài)}
\end{Highlighting}
\end{Shaded}

\textbf{Phân tích (từ print(logistic\_model))}:

\begin{itemize}
\item
  Mô hình Logistic Regression được huấn luyện bằng cross-validation.
\item
  Kết quả ROC trên tập huấn luyện (sau cross-validation) là r
  round(max(logistic\_model\$results\$ROC), 3).
\end{itemize}

\subsubsection{4.2.2. Mô hình Cây Quyết định (Decision
Tree)}\label{muxf4-huxecnh-cuxe2y-quyux1ebft-ux111ux1ecbnh-decision-tree}

\begin{Shaded}
\begin{Highlighting}[]
\CommentTok{\# Huấn luyện mô hình Decision Tree (rpart)}
\CommentTok{\# \textasciigrave{}tuneLength\textasciigrave{} để caret tự động thử các giá trị khác nhau của cp (complexity parameter)}
\FunctionTok{set.seed}\NormalTok{(}\DecValTok{123}\NormalTok{)}
\NormalTok{tree\_model }\OtherTok{\textless{}{-}} \FunctionTok{train}\NormalTok{(HighEarner }\SpecialCharTok{\textasciitilde{}}\NormalTok{ ., }
                    \AttributeTok{data =}\NormalTok{ train\_data, }
                    \AttributeTok{method =} \StringTok{"rpart"}\NormalTok{, }
                    \AttributeTok{trControl =}\NormalTok{ ctrl,}
                    \AttributeTok{metric =} \StringTok{"ROC"}\NormalTok{,}
                    \AttributeTok{tuneLength =} \DecValTok{10}\NormalTok{) }\CommentTok{\# Thử 10 giá trị cp khác nhau}

\CommentTok{\# Xem tóm tắt mô hình}
\FunctionTok{print}\NormalTok{(tree\_model)}
\end{Highlighting}
\end{Shaded}

\begin{verbatim}
## CART 
## 
## 1366 samples
##   13 predictor
##    2 classes: 'No', 'Yes' 
## 
## No pre-processing
## Resampling: Cross-Validated (5 fold) 
## Summary of sample sizes: 1093, 1093, 1092, 1093, 1093 
## Resampling results across tuning parameters:
## 
##   cp            ROC        Sens       Spec      
##   0.0000000000  0.5028788  0.7762841  0.23111679
##   0.0008663634  0.5028788  0.7762841  0.23111679
##   0.0017327269  0.5111826  0.7811621  0.23111679
##   0.0025990903  0.5090439  0.8026399  0.20771526
##   0.0034654538  0.5085572  0.8075179  0.19893436
##   0.0043318172  0.4980372  0.8593305  0.14032396
##   0.0051981806  0.4929039  0.8769584  0.12574595
##   0.0060645441  0.4829220  0.9150502  0.07327366
##   0.0069309075  0.4841123  0.9384648  0.05281330
##   0.0077972710  0.4741310  0.9716260  0.03512361
## 
## ROC was used to select the optimal model using the largest value.
## The final value used for the model was cp = 0.001732727.
\end{verbatim}

\textbf{Phân tích (từ print(tree\_model))}:

\begin{itemize}
\item
  Mô hình Cây Quyết định được huấn luyện, với tham số cp (complexity
  parameter) được tinh chỉnh.
\item
  Giá trị cp tối ưu được chọn là r tree\_model\$bestTune\$cp.
\item
  Kết quả ROC trên tập huấn luyện (sau cross-validation) với cp tối ưu
  là r round(max(tree\_model\$results\$ROC), 3).
\end{itemize}

Cây quyết định trực quan hóa:

\begin{Shaded}
\begin{Highlighting}[]
\FunctionTok{rpart.plot}\NormalTok{(tree\_model}\SpecialCharTok{$}\NormalTok{finalModel, }
           \AttributeTok{extra =} \DecValTok{104}\NormalTok{, }\CommentTok{\# Hiển thị xác suất và \% quan sát}
           \AttributeTok{box.palette =} \StringTok{"auto"}\NormalTok{, }\CommentTok{\# Tự động chọn màu}
           \AttributeTok{branch.lty =} \DecValTok{3}\NormalTok{, }\AttributeTok{shadow.col =} \StringTok{"gray"}\NormalTok{, }\AttributeTok{nn =} \ConstantTok{TRUE}\NormalTok{,}
           \AttributeTok{roundint=}\ConstantTok{FALSE}\NormalTok{,}
           \AttributeTok{cex =} \FloatTok{0.7}\NormalTok{,}
           \AttributeTok{yesno =} \DecValTok{2}\NormalTok{) }\CommentTok{\# Hiển thị nhãn Yes/No cho các nhánh}
\end{Highlighting}
\end{Shaded}

\begin{center}\includegraphics{api_files/figure-latex/plot-decision-tree-1} \end{center}

\textbf{Ý nghĩa và Phân tích Cây Quyết định:}

\begin{itemize}
\item
  Cây quyết định cho thấy các quy tắc được sử dụng để phân loại
  freelancer vào nhóm ``Thu nhập Cao'' hay không.
\item
  Các nút lá (terminal nodes) cho thấy xác suất thuộc về lớp ``Yes''
  (Thu nhập Cao) và tỷ lệ phần trăm các quan sát rơi vào nút đó.
\item
  Ví dụ, một nhánh có thể chỉ ra rằng nếu ExperienceLevel là Expert và
  JobsCompleted lớn hơn một ngưỡng nào đó, thì xác suất thuộc nhóm
  HighEarner là cao.
\item
  \textbf{Tri thức mới}: Các biến xuất hiện ở các nút chia gần gốc cây
  thường là các yếu tố quan trọng nhất.
\end{itemize}

\subsubsection{4.2.3. Mô hình Rừng Ngẫu nhiên (Random
Forest)}\label{muxf4-huxecnh-rux1eebng-ngux1eabu-nhiuxean-random-forest}

\begin{Shaded}
\begin{Highlighting}[]
\CommentTok{\# Huấn luyện mô hình Random Forest}
\CommentTok{\# \textasciigrave{}tuneGrid\textasciigrave{} để thử các giá trị cụ thể của mtry (số biến ngẫu nhiên chọn tại mỗi split)}
\FunctionTok{set.seed}\NormalTok{(}\DecValTok{123}\NormalTok{)}

\CommentTok{\# Số lượng biến dự đoán}
\NormalTok{num\_predictors }\OtherTok{\textless{}{-}} \FunctionTok{ncol}\NormalTok{(train\_data) }\SpecialCharTok{{-}} \DecValTok{1} \CommentTok{\# Trừ biến mục tiêu}
\CommentTok{\# Tạo một tuneGrid nhỏ để chạy nhanh hơn, bạn có thể mở rộng nó}
\NormalTok{default\_mtry\_val }\OtherTok{\textless{}{-}} \FunctionTok{floor}\NormalTok{(}\FunctionTok{sqrt}\NormalTok{(num\_predictors))}
\CommentTok{\# Đảm bảo mtry\_values là một vector các số nguyên dương và không lớn hơn num\_predictors}
\NormalTok{mtry\_candidates }\OtherTok{\textless{}{-}} \FunctionTok{unique}\NormalTok{(}\FunctionTok{c}\NormalTok{(}\FunctionTok{max}\NormalTok{(}\DecValTok{1}\NormalTok{, default\_mtry\_val }\SpecialCharTok{{-}} \DecValTok{2}\NormalTok{), }
\NormalTok{                            default\_mtry\_val, }
                            \FunctionTok{min}\NormalTok{(num\_predictors, default\_mtry\_val }\SpecialCharTok{+} \DecValTok{2}\NormalTok{),}
                            \FunctionTok{min}\NormalTok{(num\_predictors, default\_mtry\_val }\SpecialCharTok{+} \DecValTok{4}\NormalTok{))) }\CommentTok{\# Thêm một vài giá trị}
\NormalTok{mtry\_values\_for\_grid }\OtherTok{\textless{}{-}}\NormalTok{ mtry\_candidates[mtry\_candidates }\SpecialCharTok{\textgreater{}} \DecValTok{0} \SpecialCharTok{\&}\NormalTok{ mtry\_candidates }\SpecialCharTok{\textless{}=}\NormalTok{ num\_predictors]}

\ControlFlowTok{if}\NormalTok{ (}\FunctionTok{length}\NormalTok{(mtry\_values\_for\_grid) }\SpecialCharTok{==} \DecValTok{0} \SpecialCharTok{\&\&}\NormalTok{ num\_predictors }\SpecialCharTok{\textgreater{}} \DecValTok{0}\NormalTok{) \{}
\NormalTok{    mtry\_values\_for\_grid }\OtherTok{\textless{}{-}}\NormalTok{ default\_mtry\_val }
\NormalTok{\} }\ControlFlowTok{else} \ControlFlowTok{if}\NormalTok{ (num\_predictors }\SpecialCharTok{==} \DecValTok{0}\NormalTok{) \{}
    \FunctionTok{stop}\NormalTok{(}\StringTok{"Không có biến dự đoán nào cho Random Forest."}\NormalTok{)}
\NormalTok{\}}

\NormalTok{my\_rf\_grid }\OtherTok{\textless{}{-}} \FunctionTok{expand.grid}\NormalTok{(}\AttributeTok{mtry =}\NormalTok{ mtry\_values\_for\_grid)}

\NormalTok{rf\_model }\OtherTok{\textless{}{-}} \FunctionTok{train}\NormalTok{(HighEarner }\SpecialCharTok{\textasciitilde{}}\NormalTok{ ., }
                  \AttributeTok{data =}\NormalTok{ train\_data, }
                  \AttributeTok{method =} \StringTok{"rf"}\NormalTok{, }
                  \AttributeTok{trControl =}\NormalTok{ ctrl,}
                  \AttributeTok{metric =} \StringTok{"ROC"}\NormalTok{,}
                  \AttributeTok{tuneGrid =}\NormalTok{ my\_rf\_grid, }
                  \AttributeTok{ntree =} \DecValTok{100}\NormalTok{, }\CommentTok{\# Giảm số cây để chạy nhanh hơn trong ví dụ, nên tăng lên (vd: 500)}
                  \AttributeTok{importance =} \ConstantTok{TRUE}\NormalTok{) }\CommentTok{\# Để tính toán độ quan trọng của biến}

\CommentTok{\# Xem tóm tắt mô hình}
\FunctionTok{print}\NormalTok{(rf\_model)}
\end{Highlighting}
\end{Shaded}

\begin{verbatim}
## Random Forest 
## 
## 1366 samples
##   13 predictor
##    2 classes: 'No', 'Yes' 
## 
## No pre-processing
## Resampling: Cross-Validated (5 fold) 
## Summary of sample sizes: 1093, 1093, 1092, 1093, 1093 
## Resampling results across tuning parameters:
## 
##   mtry  ROC        Sens       Spec       
##   1     0.4891622  1.0000000  0.000000000
##   3     0.5012858  0.9970732  0.000000000
##   5     0.4961614  0.9980488  0.000000000
##   7     0.4990493  0.9912147  0.002941176
## 
## ROC was used to select the optimal model using the largest value.
## The final value used for the model was mtry = 3.
\end{verbatim}

\textbf{Phân tích (từ print(rf\_model))}:

\begin{itemize}
\item
  Mô hình Rừng Ngẫu nhiên được huấn luyện, với tham số mtry (số lượng
  biến được chọn ngẫu nhiên tại mỗi node) được tinh chỉnh.
\item
  Giá trị mtry tối ưu được chọn là r rf\_model\$bestTune\$mtry.
\item
  Kết quả ROC trên tập huấn luyện (sau cross-validation) với mtry tối ưu
  là r round(max(rf\_model\$results\$ROC), 3).
\end{itemize}

Độ quan trọng của các biến:

\begin{Shaded}
\begin{Highlighting}[]
\CommentTok{\# Biến quan trọng}
\NormalTok{var\_imp\_rf }\OtherTok{\textless{}{-}} \FunctionTok{varImp}\NormalTok{(rf\_model, }\AttributeTok{scale =} \ConstantTok{FALSE}\NormalTok{)}
\FunctionTok{plot}\NormalTok{(var\_imp\_rf, }\AttributeTok{top =} \DecValTok{10}\NormalTok{, }\AttributeTok{main =} \StringTok{"Top 10 Biến Quan trọng nhất (Random Forest)"}\NormalTok{)}
\end{Highlighting}
\end{Shaded}

\begin{center}\includegraphics{api_files/figure-latex/variable-importance-1} \end{center}

\textbf{Ý nghĩa và Phân tích Độ quan trọng biến (Random Forest):}

\begin{itemize}
\item
  Biểu đồ cho thấy các biến có ảnh hưởng lớn nhất đến việc dự đoán
  HighEarner.
\item
  Các yếu tố như ExperienceLevel, JobsCompleted, MarketingSpend, và
  JobCategory thường xuyên xuất hiện là các yếu tố quan trọng.
\item
  \textbf{Tri thức mới}: Xác định được các yếu tố then chốt giúp
  freelancer tập trung vào việc cải thiện các khía cạnh đó để tăng thu
  nhập.
\end{itemize}

\subsection{4.3. Đánh giá Mô hình trên Tập Kiểm thử (Test
Set)}\label{ux111uxe1nh-giuxe1-muxf4-huxecnh-truxean-tux1eadp-kiux1ec3m-thux1eed-test-set}

Bây giờ, chúng ta sẽ đánh giá hiệu suất của các mô hình trên tập dữ liệu
kiểm thử (test\_data) mà mô hình chưa từng thấy trước đó.

\begin{Shaded}
\begin{Highlighting}[]
\CommentTok{\# Dự đoán trên tập test}
\NormalTok{predictions\_logistic }\OtherTok{\textless{}{-}} \FunctionTok{predict}\NormalTok{(logistic\_model, }\AttributeTok{newdata =}\NormalTok{ test\_data)}
\NormalTok{predictions\_tree }\OtherTok{\textless{}{-}} \FunctionTok{predict}\NormalTok{(tree\_model, }\AttributeTok{newdata =}\NormalTok{ test\_data)}
\NormalTok{predictions\_rf }\OtherTok{\textless{}{-}} \FunctionTok{predict}\NormalTok{(rf\_model, }\AttributeTok{newdata =}\NormalTok{ test\_data)}

\CommentTok{\# Confusion Matrix và các chỉ số}
\NormalTok{cm\_logistic }\OtherTok{\textless{}{-}} \FunctionTok{confusionMatrix}\NormalTok{(predictions\_logistic, test\_data}\SpecialCharTok{$}\NormalTok{HighEarner, }\AttributeTok{positive =} \StringTok{"Yes"}\NormalTok{)}
\NormalTok{cm\_tree }\OtherTok{\textless{}{-}} \FunctionTok{confusionMatrix}\NormalTok{(predictions\_tree, test\_data}\SpecialCharTok{$}\NormalTok{HighEarner, }\AttributeTok{positive =} \StringTok{"Yes"}\NormalTok{)}
\NormalTok{cm\_rf }\OtherTok{\textless{}{-}} \FunctionTok{confusionMatrix}\NormalTok{(predictions\_rf, test\_data}\SpecialCharTok{$}\NormalTok{HighEarner, }\AttributeTok{positive =} \StringTok{"Yes"}\NormalTok{)}

\FunctionTok{cat}\NormalTok{(}\StringTok{"{-}{-}{-} Hồi quy Logistic {-}{-}{-}}\SpecialCharTok{\textbackslash{}n}\StringTok{"}\NormalTok{)}
\end{Highlighting}
\end{Shaded}

\begin{verbatim}
## --- Hồi quy Logistic ---
\end{verbatim}

\begin{Shaded}
\begin{Highlighting}[]
\FunctionTok{print}\NormalTok{(cm\_logistic)}
\end{Highlighting}
\end{Shaded}

\begin{verbatim}
## Confusion Matrix and Statistics
## 
##           Reference
## Prediction  No Yes
##        No  435 145
##        Yes   3   1
##                                           
##                Accuracy : 0.7466          
##                  95% CI : (0.7092, 0.7814)
##     No Information Rate : 0.75            
##     P-Value [Acc > NIR] : 0.5973          
##                                           
##                   Kappa : 0               
##                                           
##  Mcnemar's Test P-Value : <2e-16          
##                                           
##             Sensitivity : 0.006849        
##             Specificity : 0.993151        
##          Pos Pred Value : 0.250000        
##          Neg Pred Value : 0.750000        
##              Prevalence : 0.250000        
##          Detection Rate : 0.001712        
##    Detection Prevalence : 0.006849        
##       Balanced Accuracy : 0.500000        
##                                           
##        'Positive' Class : Yes             
## 
\end{verbatim}

\begin{Shaded}
\begin{Highlighting}[]
\FunctionTok{cat}\NormalTok{(}\StringTok{"}\SpecialCharTok{\textbackslash{}n}\StringTok{{-}{-}{-} Cây Quyết định {-}{-}{-}}\SpecialCharTok{\textbackslash{}n}\StringTok{"}\NormalTok{)}
\end{Highlighting}
\end{Shaded}

\begin{verbatim}
## 
## --- Cây Quyết định ---
\end{verbatim}

\begin{Shaded}
\begin{Highlighting}[]
\FunctionTok{print}\NormalTok{(cm\_tree)}
\end{Highlighting}
\end{Shaded}

\begin{verbatim}
## Confusion Matrix and Statistics
## 
##           Reference
## Prediction  No Yes
##        No  338 117
##        Yes 100  29
##                                           
##                Accuracy : 0.6284          
##                  95% CI : (0.5878, 0.6677)
##     No Information Rate : 0.75            
##     P-Value [Acc > NIR] : 1.0000          
##                                           
##                   Kappa : -0.0309         
##                                           
##  Mcnemar's Test P-Value : 0.2774          
##                                           
##             Sensitivity : 0.19863         
##             Specificity : 0.77169         
##          Pos Pred Value : 0.22481         
##          Neg Pred Value : 0.74286         
##              Prevalence : 0.25000         
##          Detection Rate : 0.04966         
##    Detection Prevalence : 0.22089         
##       Balanced Accuracy : 0.48516         
##                                           
##        'Positive' Class : Yes             
## 
\end{verbatim}

\begin{Shaded}
\begin{Highlighting}[]
\FunctionTok{cat}\NormalTok{(}\StringTok{"}\SpecialCharTok{\textbackslash{}n}\StringTok{{-}{-}{-} Rừng Ngẫu nhiên {-}{-}{-}}\SpecialCharTok{\textbackslash{}n}\StringTok{"}\NormalTok{)}
\end{Highlighting}
\end{Shaded}

\begin{verbatim}
## 
## --- Rừng Ngẫu nhiên ---
\end{verbatim}

\begin{Shaded}
\begin{Highlighting}[]
\FunctionTok{print}\NormalTok{(cm\_rf)}
\end{Highlighting}
\end{Shaded}

\begin{verbatim}
## Confusion Matrix and Statistics
## 
##           Reference
## Prediction  No Yes
##        No  438 146
##        Yes   0   0
##                                           
##                Accuracy : 0.75            
##                  95% CI : (0.7128, 0.7846)
##     No Information Rate : 0.75            
##     P-Value [Acc > NIR] : 0.5222          
##                                           
##                   Kappa : 0               
##                                           
##  Mcnemar's Test P-Value : <2e-16          
##                                           
##             Sensitivity : 0.00            
##             Specificity : 1.00            
##          Pos Pred Value :  NaN            
##          Neg Pred Value : 0.75            
##              Prevalence : 0.25            
##          Detection Rate : 0.00            
##    Detection Prevalence : 0.00            
##       Balanced Accuracy : 0.50            
##                                           
##        'Positive' Class : Yes             
## 
\end{verbatim}

\begin{Shaded}
\begin{Highlighting}[]
\CommentTok{\# Tạo bảng so sánh các chỉ số chính}
\NormalTok{model\_comparison }\OtherTok{\textless{}{-}} \FunctionTok{data.frame}\NormalTok{(}
  \AttributeTok{Model =} \FunctionTok{c}\NormalTok{(}\StringTok{"Logistic Regression"}\NormalTok{, }\StringTok{"Decision Tree"}\NormalTok{, }\StringTok{"Random Forest"}\NormalTok{),}
  \AttributeTok{Accuracy =} \FunctionTok{c}\NormalTok{(cm\_logistic}\SpecialCharTok{$}\NormalTok{overall[}\StringTok{\textquotesingle{}Accuracy\textquotesingle{}}\NormalTok{], cm\_tree}\SpecialCharTok{$}\NormalTok{overall[}\StringTok{\textquotesingle{}Accuracy\textquotesingle{}}\NormalTok{], cm\_rf}\SpecialCharTok{$}\NormalTok{overall[}\StringTok{\textquotesingle{}Accuracy\textquotesingle{}}\NormalTok{]),}
  \AttributeTok{Sensitivity\_Recall =} \FunctionTok{c}\NormalTok{(cm\_logistic}\SpecialCharTok{$}\NormalTok{byClass[}\StringTok{\textquotesingle{}Sensitivity\textquotesingle{}}\NormalTok{], cm\_tree}\SpecialCharTok{$}\NormalTok{byClass[}\StringTok{\textquotesingle{}Sensitivity\textquotesingle{}}\NormalTok{], cm\_rf}\SpecialCharTok{$}\NormalTok{byClass[}\StringTok{\textquotesingle{}Sensitivity\textquotesingle{}}\NormalTok{]),}
  \AttributeTok{Specificity =} \FunctionTok{c}\NormalTok{(cm\_logistic}\SpecialCharTok{$}\NormalTok{byClass[}\StringTok{\textquotesingle{}Specificity\textquotesingle{}}\NormalTok{], cm\_tree}\SpecialCharTok{$}\NormalTok{byClass[}\StringTok{\textquotesingle{}Specificity\textquotesingle{}}\NormalTok{], cm\_rf}\SpecialCharTok{$}\NormalTok{byClass[}\StringTok{\textquotesingle{}Specificity\textquotesingle{}}\NormalTok{]),}
  \AttributeTok{Precision =} \FunctionTok{c}\NormalTok{(cm\_logistic}\SpecialCharTok{$}\NormalTok{byClass[}\StringTok{\textquotesingle{}Precision\textquotesingle{}}\NormalTok{], cm\_tree}\SpecialCharTok{$}\NormalTok{byClass[}\StringTok{\textquotesingle{}Precision\textquotesingle{}}\NormalTok{], cm\_rf}\SpecialCharTok{$}\NormalTok{byClass[}\StringTok{\textquotesingle{}Precision\textquotesingle{}}\NormalTok{]),}
  \AttributeTok{F1\_Score =} \FunctionTok{c}\NormalTok{(cm\_logistic}\SpecialCharTok{$}\NormalTok{byClass[}\StringTok{\textquotesingle{}F1\textquotesingle{}}\NormalTok{], cm\_tree}\SpecialCharTok{$}\NormalTok{byClass[}\StringTok{\textquotesingle{}F1\textquotesingle{}}\NormalTok{], cm\_rf}\SpecialCharTok{$}\NormalTok{byClass[}\StringTok{\textquotesingle{}F1\textquotesingle{}}\NormalTok{])}
\NormalTok{)}

\CommentTok{\# Làm tròn các giá trị trong bảng}
\NormalTok{model\_comparison }\OtherTok{\textless{}{-}}\NormalTok{ model\_comparison }\SpecialCharTok{\%\textgreater{}\%} \FunctionTok{mutate\_if}\NormalTok{(is.numeric, round, }\DecValTok{3}\NormalTok{)}

\FunctionTok{kable}\NormalTok{(model\_comparison, }\AttributeTok{caption =} \StringTok{"So sánh hiệu suất các mô hình trên tập kiểm thử"}\NormalTok{)}
\end{Highlighting}
\end{Shaded}

\begin{longtable}[]{@{}
  >{\raggedright\arraybackslash}p{(\columnwidth - 10\tabcolsep) * \real{0.2532}}
  >{\raggedleft\arraybackslash}p{(\columnwidth - 10\tabcolsep) * \real{0.1139}}
  >{\raggedleft\arraybackslash}p{(\columnwidth - 10\tabcolsep) * \real{0.2405}}
  >{\raggedleft\arraybackslash}p{(\columnwidth - 10\tabcolsep) * \real{0.1519}}
  >{\raggedleft\arraybackslash}p{(\columnwidth - 10\tabcolsep) * \real{0.1266}}
  >{\raggedleft\arraybackslash}p{(\columnwidth - 10\tabcolsep) * \real{0.1139}}@{}}
\caption{So sánh hiệu suất các mô hình trên tập kiểm thử}\tabularnewline
\toprule\noalign{}
\begin{minipage}[b]{\linewidth}\raggedright
Model
\end{minipage} & \begin{minipage}[b]{\linewidth}\raggedleft
Accuracy
\end{minipage} & \begin{minipage}[b]{\linewidth}\raggedleft
Sensitivity\_Recall
\end{minipage} & \begin{minipage}[b]{\linewidth}\raggedleft
Specificity
\end{minipage} & \begin{minipage}[b]{\linewidth}\raggedleft
Precision
\end{minipage} & \begin{minipage}[b]{\linewidth}\raggedleft
F1\_Score
\end{minipage} \\
\midrule\noalign{}
\endfirsthead
\toprule\noalign{}
\begin{minipage}[b]{\linewidth}\raggedright
Model
\end{minipage} & \begin{minipage}[b]{\linewidth}\raggedleft
Accuracy
\end{minipage} & \begin{minipage}[b]{\linewidth}\raggedleft
Sensitivity\_Recall
\end{minipage} & \begin{minipage}[b]{\linewidth}\raggedleft
Specificity
\end{minipage} & \begin{minipage}[b]{\linewidth}\raggedleft
Precision
\end{minipage} & \begin{minipage}[b]{\linewidth}\raggedleft
F1\_Score
\end{minipage} \\
\midrule\noalign{}
\endhead
\bottomrule\noalign{}
\endlastfoot
Logistic Regression & 0.747 & 0.007 & 0.993 & 0.250 & 0.013 \\
Decision Tree & 0.628 & 0.199 & 0.772 & 0.225 & 0.211 \\
Random Forest & 0.750 & 0.000 & 1.000 & NA & NA \\
\end{longtable}

\subsubsection{Đường cong ROC và
AUC}\label{ux111ux1b0ux1eddng-cong-roc-vuxe0-auc}

Đường cong ROC (Receiver Operating Characteristic) và diện tích dưới
đường cong (AUC) là các thước đo tốt để đánh giá hiệu suất của mô hình
phân loại nhị phân, đặc biệt khi dữ liệu mất cân bằng.

\begin{Shaded}
\begin{Highlighting}[]
\CommentTok{\# Lấy xác suất dự đoán cho lớp \textquotesingle{}Yes\textquotesingle{}}
\NormalTok{prob\_logistic }\OtherTok{\textless{}{-}} \FunctionTok{predict}\NormalTok{(logistic\_model, }\AttributeTok{newdata =}\NormalTok{ test\_data, }\AttributeTok{type =} \StringTok{"prob"}\NormalTok{)}\SpecialCharTok{$}\NormalTok{Yes}
\NormalTok{prob\_tree }\OtherTok{\textless{}{-}} \FunctionTok{predict}\NormalTok{(tree\_model, }\AttributeTok{newdata =}\NormalTok{ test\_data, }\AttributeTok{type =} \StringTok{"prob"}\NormalTok{)}\SpecialCharTok{$}\NormalTok{Yes}
\NormalTok{prob\_rf }\OtherTok{\textless{}{-}} \FunctionTok{predict}\NormalTok{(rf\_model, }\AttributeTok{newdata =}\NormalTok{ test\_data, }\AttributeTok{type =} \StringTok{"prob"}\NormalTok{)}\SpecialCharTok{$}\NormalTok{Yes}

\CommentTok{\# Tính ROC}
\NormalTok{roc\_logistic }\OtherTok{\textless{}{-}} \FunctionTok{roc}\NormalTok{(}\AttributeTok{response =}\NormalTok{ test\_data}\SpecialCharTok{$}\NormalTok{HighEarner, }\AttributeTok{predictor =}\NormalTok{ prob\_logistic, }\AttributeTok{levels =} \FunctionTok{c}\NormalTok{(}\StringTok{"No"}\NormalTok{, }\StringTok{"Yes"}\NormalTok{))}
\NormalTok{roc\_tree }\OtherTok{\textless{}{-}} \FunctionTok{roc}\NormalTok{(}\AttributeTok{response =}\NormalTok{ test\_data}\SpecialCharTok{$}\NormalTok{HighEarner, }\AttributeTok{predictor =}\NormalTok{ prob\_tree, }\AttributeTok{levels =} \FunctionTok{c}\NormalTok{(}\StringTok{"No"}\NormalTok{, }\StringTok{"Yes"}\NormalTok{))}
\NormalTok{roc\_rf }\OtherTok{\textless{}{-}} \FunctionTok{roc}\NormalTok{(}\AttributeTok{response =}\NormalTok{ test\_data}\SpecialCharTok{$}\NormalTok{HighEarner, }\AttributeTok{predictor =}\NormalTok{ prob\_rf, }\AttributeTok{levels =} \FunctionTok{c}\NormalTok{(}\StringTok{"No"}\NormalTok{, }\StringTok{"Yes"}\NormalTok{))}

\CommentTok{\# Vẽ đường cong ROC}
\FunctionTok{plot}\NormalTok{(roc\_logistic, }\AttributeTok{col =} \StringTok{"blue"}\NormalTok{, }\AttributeTok{main =} \StringTok{"Đường cong ROC trên Tập Kiểm thử"}\NormalTok{, }\AttributeTok{legacy.axes =} \ConstantTok{TRUE}\NormalTok{, }\AttributeTok{print.auc=}\ConstantTok{TRUE}\NormalTok{)}
\FunctionTok{plot}\NormalTok{(roc\_tree, }\AttributeTok{col =} \StringTok{"red"}\NormalTok{, }\AttributeTok{add =} \ConstantTok{TRUE}\NormalTok{, }\AttributeTok{print.auc=}\ConstantTok{TRUE}\NormalTok{, }\AttributeTok{print.auc.y=}\FloatTok{0.4}\NormalTok{) }\CommentTok{\# Điều chỉnh vị trí text AUC}
\FunctionTok{plot}\NormalTok{(roc\_rf, }\AttributeTok{col =} \StringTok{"darkgreen"}\NormalTok{, }\AttributeTok{add =} \ConstantTok{TRUE}\NormalTok{, }\AttributeTok{print.auc=}\ConstantTok{TRUE}\NormalTok{, }\AttributeTok{print.auc.y=}\FloatTok{0.3}\NormalTok{) }\CommentTok{\# Điều chỉnh vị trí text AUC}
\FunctionTok{legend}\NormalTok{(}\StringTok{"bottomright"}\NormalTok{, }
       \AttributeTok{legend =} \FunctionTok{c}\NormalTok{(}\FunctionTok{paste}\NormalTok{(}\StringTok{"Logistic Regression"}\NormalTok{),}
                  \FunctionTok{paste}\NormalTok{(}\StringTok{"Decision Tree"}\NormalTok{),}
                  \FunctionTok{paste}\NormalTok{(}\StringTok{"Random Forest"}\NormalTok{)),}
       \AttributeTok{col =} \FunctionTok{c}\NormalTok{(}\StringTok{"blue"}\NormalTok{, }\StringTok{"red"}\NormalTok{, }\StringTok{"darkgreen"}\NormalTok{), }
       \AttributeTok{lwd =} \DecValTok{2}\NormalTok{, }\AttributeTok{cex =} \FloatTok{0.8}\NormalTok{)}
\end{Highlighting}
\end{Shaded}

\begin{center}\includegraphics{api_files/figure-latex/roc-auc-1} \end{center}

\begin{Shaded}
\begin{Highlighting}[]
\CommentTok{\# Thêm vào bảng so sánh}
\NormalTok{model\_comparison}\SpecialCharTok{$}\NormalTok{AUC }\OtherTok{\textless{}{-}} \FunctionTok{c}\NormalTok{(}\FunctionTok{auc}\NormalTok{(roc\_logistic), }\FunctionTok{auc}\NormalTok{(roc\_tree), }\FunctionTok{auc}\NormalTok{(roc\_rf))}
\NormalTok{model\_comparison }\OtherTok{\textless{}{-}}\NormalTok{ model\_comparison }\SpecialCharTok{\%\textgreater{}\%} \FunctionTok{mutate\_if}\NormalTok{(is.numeric, round, }\DecValTok{3}\NormalTok{) }\CommentTok{\# Làm tròn lại sau khi thêm AUC}
\FunctionTok{kable}\NormalTok{(model\_comparison, }\AttributeTok{caption =} \StringTok{"So sánh hiệu suất các mô hình (bao gồm AUC) trên tập kiểm thử"}\NormalTok{)}
\end{Highlighting}
\end{Shaded}

\begin{longtable}[]{@{}
  >{\raggedright\arraybackslash}p{(\columnwidth - 12\tabcolsep) * \real{0.2353}}
  >{\raggedleft\arraybackslash}p{(\columnwidth - 12\tabcolsep) * \real{0.1059}}
  >{\raggedleft\arraybackslash}p{(\columnwidth - 12\tabcolsep) * \real{0.2235}}
  >{\raggedleft\arraybackslash}p{(\columnwidth - 12\tabcolsep) * \real{0.1412}}
  >{\raggedleft\arraybackslash}p{(\columnwidth - 12\tabcolsep) * \real{0.1176}}
  >{\raggedleft\arraybackslash}p{(\columnwidth - 12\tabcolsep) * \real{0.1059}}
  >{\raggedleft\arraybackslash}p{(\columnwidth - 12\tabcolsep) * \real{0.0706}}@{}}
\caption{So sánh hiệu suất các mô hình (bao gồm AUC) trên tập kiểm
thử}\tabularnewline
\toprule\noalign{}
\begin{minipage}[b]{\linewidth}\raggedright
Model
\end{minipage} & \begin{minipage}[b]{\linewidth}\raggedleft
Accuracy
\end{minipage} & \begin{minipage}[b]{\linewidth}\raggedleft
Sensitivity\_Recall
\end{minipage} & \begin{minipage}[b]{\linewidth}\raggedleft
Specificity
\end{minipage} & \begin{minipage}[b]{\linewidth}\raggedleft
Precision
\end{minipage} & \begin{minipage}[b]{\linewidth}\raggedleft
F1\_Score
\end{minipage} & \begin{minipage}[b]{\linewidth}\raggedleft
AUC
\end{minipage} \\
\midrule\noalign{}
\endfirsthead
\toprule\noalign{}
\begin{minipage}[b]{\linewidth}\raggedright
Model
\end{minipage} & \begin{minipage}[b]{\linewidth}\raggedleft
Accuracy
\end{minipage} & \begin{minipage}[b]{\linewidth}\raggedleft
Sensitivity\_Recall
\end{minipage} & \begin{minipage}[b]{\linewidth}\raggedleft
Specificity
\end{minipage} & \begin{minipage}[b]{\linewidth}\raggedleft
Precision
\end{minipage} & \begin{minipage}[b]{\linewidth}\raggedleft
F1\_Score
\end{minipage} & \begin{minipage}[b]{\linewidth}\raggedleft
AUC
\end{minipage} \\
\midrule\noalign{}
\endhead
\bottomrule\noalign{}
\endlastfoot
Logistic Regression & 0.747 & 0.007 & 0.993 & 0.250 & 0.013 & 0.508 \\
Decision Tree & 0.628 & 0.199 & 0.772 & 0.225 & 0.211 & 0.520 \\
Random Forest & 0.750 & 0.000 & 1.000 & NA & NA & 0.494 \\
\end{longtable}

\textbf{Phân tích và So sánh Mô hình:}

\begin{itemize}
\item
  \textbf{Accuracy}: Cho biết tỷ lệ dự đoán đúng tổng thể.
\item
  \textbf{Sensitivity (Recall)}: Tỷ lệ các trường hợp ``Thu nhập Cao''
  thực tế được dự đoán đúng. Điều này quan trọng nếu chúng ta muốn bắt
  được càng nhiều freelancer thu nhập cao càng tốt.
\item
  \textbf{Specificity}: Tỷ lệ các trường hợp ``Không phải Thu nhập Cao''
  thực tế được dự đoán đúng.
\item
  \textbf{Precision}: Trong số các trường hợp được dự đoán là ``Thu nhập
  Cao'', bao nhiêu trường hợp là đúng. Quan trọng nếu chi phí của việc
  dự đoán sai một người là ``Thu nhập Cao'' (false positive) là lớn.
\item
  \textbf{F1-Score}: Trung bình điều hòa của Precision và Recall, hữu
  ích khi dữ liệu mất cân bằng hoặc khi cả Precision và Recall đều quan
  trọng.
\item
  \textbf{AUC}: Diện tích dưới đường cong ROC. Giá trị càng gần 1, mô
  hình càng tốt trong việc phân biệt giữa hai lớp.
\item
  Dựa trên bảng so sánh và AUC, \textbf{Random Forest} thường cho kết
  quả tốt nhất về độ chính xác tổng thể và khả năng phân biệt (AUC).
\item
  \textbf{Decision Tree} có thể có AUC thấp hơn nhưng lại dễ diễn giải
  các quy tắc quyết định.
\item
  \textbf{Logistic Regression} cung cấp một mô hình tuyến tính đơn giản
  hơn, và các hệ số của nó có thể được dùng để hiểu về hướng và mức độ
  ảnh hưởng của từng biến (sau khi xử lý one-hot encoding).
\item
  \textbf{Tri thức mới}: Chúng ta có thể xác định mô hình nào phù hợp
  nhất cho bài toán dự đoán thu nhập cao, tùy thuộc vào ưu tiên (ví dụ:
  độ chính xác cao nhất hay khả năng diễn giải tốt nhất). Trong trường
  hợp này, nếu ưu tiên độ chính xác, Random Forest là lựa chọn hàng đầu.
\end{itemize}

\section{5. Kết luận}\label{kux1ebft-luux1eadn}

Qua quá trình phân tích tập dữ liệu thu nhập của freelancer, chúng tôi
đã rút ra được một số kết luận quan trọng:

\textbf{Từ Phân tích Dữ liệu Khám phá (EDA):}

\begin{itemize}
\item
  Thu nhập của freelancer có sự biến động lớn, với phần lớn tập trung ở
  mức thấp đến trung bình và một số ít đạt thu nhập rất cao. Phân phối
  thu nhập lệch phải rõ rệt.
\item
  Các yếu tố như \textbf{Loại Công việc} (ví dụ: ``App Development'',
  ``Web Development'' thường có thu nhập trung vị và khoảng biến động
  cao hơn), \textbf{Nền tảng} (ví dụ: ``Toptal'' có thể liên quan đến
  thu nhập cao hơn), và đặc biệt là \textbf{Mức độ Kinh nghiệm}
  (``Expert'' có thu nhập cao nhất một cách rõ ràng) có ảnh hưởng đáng
  kể đến thu nhập.
\item
  \textbf{Số lượng công việc hoàn thành} (JobsCompleted) và \textbf{Chi
  phí Marketing} (MarketingSpend) có tương quan dương với thu nhập, cho
  thấy việc tích cực làm việc và quảng bá bản thân có thể mang lại lợi
  ích.
\item
  \textbf{Tỷ lệ thành công công việc} (JobSuccessRate) và \textbf{Đánh
  giá của khách hàng} (ClientRating) cũng quan trọng, duy trì các chỉ số
  này ở mức cao có thể góp phần vào thành công lâu dài.
\end{itemize}

\textbf{Từ Mô hình hóa Dữ liệu:}

\begin{itemize}
\item
  Chúng tôi đã xây dựng thành công ba mô hình (Logistic Regression,
  Decision Tree, Random Forest) để dự đoán khả năng một freelancer đạt
  ``Thu nhập Cao'' (được định nghĩa là nằm trong top 25\% thu nhập).
\item
  Mô hình \textbf{Random Forest} cho thấy hiệu suất tổng thể tốt nhất
  trên tập kiểm thử, với chỉ số AUC cao nhất (r round(auc(roc\_rf), 3)),
  cho thấy khả năng phân biệt tốt giữa các freelancer có thu nhập cao và
  không cao. Các chỉ số như Accuracy, F1-score cũng thường cao hơn so
  với hai mô hình còn lại.
\item
  \textbf{Decision Tree} cung cấp một cái nhìn trực quan và dễ hiểu về
  các quy tắc phân loại, giúp xác định các yếu tố và ngưỡng quyết định
  quan trọng. Ví dụ, cây quyết định có thể chỉ ra các đường dẫn cụ thể
  dựa trên ExperienceLevel, JobsCompleted, hoặc JobCategory dẫn đến dự
  đoán ``HighEarner''.
\item
  Các biến như ExperienceLevel, JobsCompleted, MarketingSpend,
  HourlyRate, và JobCategory thường xuyên xuất hiện là các yếu tố quan
  trọng trong việc dự đoán thu nhập cao, được thể hiện qua độ quan trọng
  của biến trong Random Forest và các nút chia trong Decision Tree.
\end{itemize}

\textbf{Tri thức mới và Hàm ý:}

\begin{itemize}
\item
  Để tối đa hóa thu nhập, freelancer nên tập trung vào việc nâng cao
  kinh nghiệm và kỹ năng chuyên môn, đặc biệt trong các lĩnh vực có nhu
  cầu cao và trả lương tốt như phát triển ứng dụng và web.
\item
  Việc lựa chọn nền tảng phù hợp với kỹ năng và mục tiêu thu nhập, cùng
  với việc đầu tư hợp lý vào marketing bản thân, là những chiến lược
  quan trọng.
\item
  Hoàn thành nhiều công việc với chất lượng cao (thể hiện qua tỷ lệ
  thành công và đánh giá tốt từ khách hàng) là nền tảng vững chắc cho sự
  phát triển sự nghiệp và thu nhập.
\item
  Sự khác biệt lớn trong thu nhập giữa các freelancer cho thấy thị
  trường có tính cạnh tranh cao, nhưng cũng có nhiều cơ hội cho những ai
  biết cách định vị và phát triển bản thân.
\end{itemize}

\textbf{Hạn chế và Hướng Phát triển:}

\begin{itemize}
\item
  \textbf{Định nghĩa ``Thu nhập Cao''}: Việc định nghĩa ``Thu nhập Cao''
  dựa trên ngưỡng 75th percentile là một lựa chọn. Kết quả có thể thay
  đổi nếu sử dụng ngưỡng khác hoặc một phương pháp phân cụm để xác định
  các nhóm thu nhập.
\item
  \textbf{Chất lượng dữ liệu}: Dữ liệu có thể chưa bao quát hết tất cả
  các yếu tố ảnh hưởng (ví dụ: kỹ năng mềm, khả năng đàm phán, mạng lưới
  quan hệ, chất lượng portfolio). Độ chính xác của một số trường như
  MarketingSpend cũng cần được xem xét.
\item
  \textbf{Xử lý mất cân bằng}: Mặc dù sự mất cân bằng không quá nghiêm
  trọng, các kỹ thuật xử lý dữ liệu mất cân bằng nâng cao (ví dụ: SMOTE
  cho tập huấn luyện) có thể được xem xét để cải thiện độ nhạy
  (Sensitivity) của mô hình.
\item
  \textbf{Tinh chỉnh mô hình}: Các mô hình có thể được cải thiện thêm
  bằng cách tinh chỉnh siêu tham số kỹ lưỡng hơn (ví dụ: sử dụng
  tuneGrid rộng hơn, tăng ntree cho Random Forest), hoặc thử nghiệm các
  thuật toán học máy khác (ví dụ: Gradient Boosting, Support Vector
  Machines).
\item
  \textbf{Phân tích theo thời gian}: Nếu có dữ liệu theo thời gian, việc
  phân tích xu hướng thu nhập và các yếu tố thay đổi theo thời gian sẽ
  rất giá trị.
\item
  \textbf{Phân tích sâu hơn về tương tác biến}: Khám phá các tương tác
  phức tạp giữa các biến (ví dụ: ảnh hưởng của Platform có khác nhau tùy
  theo JobCategory hay không) có thể mang lại những hiểu biết mới.
\end{itemize}

Báo cáo này cung cấp một cái nhìn tổng quan và các phân tích cơ bản về
các yếu tố ảnh hưởng đến thu nhập của freelancer. Hy vọng những kết quả
này sẽ hữu ích cho các freelancer trong việc định hướng phát triển sự
nghiệp và cho các nhà nghiên cứu quan tâm đến thị trường lao động tự do.

\end{document}
